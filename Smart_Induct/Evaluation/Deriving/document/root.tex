\documentclass[11pt,a4paper]{article}
\usepackage{isabelle,isabellesym}

% this should be the last package used
\usepackage{pdfsetup}
\usepackage{railsetup}

% urls in roman style, theory text in math-similar italics
\urlstyle{rm}
\isabellestyle{it}

\newcommand\isafor{\textsf{IsaFoR}}
\newcommand\ceta{\textsf{Ce\kern-.18emT\kern-.18emA}}

\begin{document}

\title{Deriving class instances for datatypes.\footnote{Supported by FWF (Austrian Science Fund) projects P27502 and Y757.}}
\author{Christian Sternagel and Ren\'e Thiemann}
\maketitle

\begin{abstract}
  We provide a framework for registering automatic methods 
  to derive class instances 
  of datatypes, 
  as it is possible using Haskell's ``deriving Ord, Show, \ldots'' feature.
  
  We further implemented such automatic methods to derive comparators, linear orders, parametrizable equality functions,
  and hash-functions which are required in the 
  Isabelle Collection Framework \cite{rbt} and the Container Framework \cite{containers}. 
  Moreover, for the tactic of Blanchette to show that a datatype is countable, we implemented a 
  wrapper so that this tactic becomes accessible in our framework. All of the generators are based on 
  the infrastructure that is provided by the BNF-based datatype package.
  
  Our formalization was performed as part of the \isafor/\ceta{} project%
  \footnote{\url{http://cl-informatik.uibk.ac.at/software/ceta}} \cite{CeTA}.
  With our new tactics we could remove 
  several tedious proofs for (conditional) linear orders, and conditional equality operators
  within \isafor{} and the Container Framework.
\end{abstract}

\tableofcontents


% include generated text of all theories
\input{session}

\section{Acknowledgements}
We thank 
\begin{itemize}
\item Lukas Bulwahn and Brian Huffman for the discussion on a generic derive command.
\item Jasmin Blanchette for providing the tactic for countability for BNF-based datatypes.
\item Jasmin Blanchette and Dmitriy Traytel for adjusting the Isabelle/ML interface of
  the BNF-based datatypes.
\item Alexander Krauss for telling us to avoid the function package for this task.
\item Peter Lammich for the inspiration of developing a hash-function generator.
\item Andreas Lochbihler for the inspiration of developing generators for the container framework.
\item Christian Urban for his cookbook on Isabelle/ML.
\item Stefan Berghofer, Florian Haftmann, Cezary Kaliszyk, Tobias Nipkow, and Makarius Wenzel for their explanations
  on several Isabelle related questions.
\end{itemize}

\bibliographystyle{abbrv}
\bibliography{root}

\end{document}
