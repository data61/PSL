\documentclass[11pt,a4paper]{article}
\usepackage{isabelle,isabellesym}

% further packages required for unusual symbols (see also
% isabellesym.sty), use only when needed

%\usepackage{amssymb}
  %for \<leadsto>, \<box>, \<diamond>, \<sqsupset>, \<mho>, \<Join>,
  %\<lhd>, \<lesssim>, \<greatersim>, \<lessapprox>, \<greaterapprox>,
  %\<triangleq>, \<yen>, \<lozenge>

%\usepackage{eurosym}
  %for \<euro>

%\usepackage[only,bigsqcap]{stmaryrd}
  %for \<Sqinter>

%\usepackage{eufrak}
  %for \<AA> ... \<ZZ>, \<aa> ... \<zz> (also included in amssymb)

%\usepackage{textcomp}
  %for \<onequarter>, \<onehalf>, \<threequarters>, \<degree>, \<cent>,
  %\<currency>

% this should be the last package used
\usepackage{pdfsetup}

% urls in roman style, theory text in math-similar italics
\urlstyle{rm}
\isabellestyle{it}

% for uniform font size
%\renewcommand{\isastyle}{\isastyleminor}


\begin{document}

\title{Authenticated Data Structures as Functors}
\author{Andreas Lochbihler \qquad Ognjen Maric \\[1em] Digital Asset}

\maketitle

\begin{abstract}
  Authenticated data structures allow several systems to convince each other that they are referring to the same data structure,
  even if each of them knows only a part of the data structure.
  Using inclusion proofs, knowledgable systems can selectively share their knowledge with other systems
  and the latter can verify the authenticity of what is being shared.

  In this paper, we show how to modularly define authenticated data structures, their inclusion proofs, and operations thereon
  as datatypes in Isabelle/HOL, using a shallow embedding.
  Modularity allows us to construct complicated trees from reusable building blocks, which we call Merkle functors.
  Merkle functors include sums, products, and function spaces and are closed under composition and least fixpoints.

  As a practical application, we model the hierarchical transactions of Canton,
  a practical interoperability protocol for distributed ledgers, as authenticated data structures.
  This is a first step towards formalizing the Canton protocol and verifying its integrity and security guarantees.
\end{abstract}


\tableofcontents

% sane default for proof documents
\parindent 0pt\parskip 0.5ex

% generated text of all theories
\input{session}

% optional bibliography
%\bibliographystyle{abbrv}
%\bibliography{root}

\end{document}

%%% Local Variables:
%%% mode: latex
%%% TeX-master: t
%%% End:
