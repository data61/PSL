\documentclass[11pt,a4paper]{article}
\usepackage{isabelle,isabellesym}

\newcommand{\doi}[1]{doi:\href{https://dx.doi.org/#1}{#1}}

% further packages required for unusual symbols (see also
% isabellesym.sty), use only when needed

\usepackage{amsmath}

%\usepackage{amssymb}
  %for \<leadsto>, \<box>, \<diamond>, \<sqsupset>, \<mho>, \<Join>,
  %\<lhd>, \<lesssim>, \<greatersim>, \<lessapprox>, \<greaterapprox>,
  %\<triangleq>, \<yen>, \<lozenge>

%\usepackage{eurosym}
  %for \<euro>

%\usepackage[only,bigsqcap]{stmaryrd}
  %for \<Sqinter>

%\usepackage{eufrak}
  %for \<AA> ... \<ZZ>, \<aa> ... \<zz> (also included in amssymb)

%\usepackage{textcomp}
  %for \<onequarter>, \<onehalf>, \<threequarters>, \<degree>, \<cent>,
  %\<currency>

% this should be the last package used
\usepackage{pdfsetup}

% urls in roman style, theory text in math-similar italics
\urlstyle{rm}
\isabellestyle{it}

% for uniform font size
%\renewcommand{\isastyle}{\isastyleminor}


\begin{document}

\title{Implementing the Goodstein Function in $\lambda$-Calculus}
\author{Bertram Felgenhauer}
\maketitle

\begin{abstract}
In this formalization, we develop an implementation of the Goodstein
function $\mathcal{G}$ in plain $\lambda$-calculus, linked to a
concise, self-contained specification. The implementation works on a
Church-encoded representation of countable ordinals. The initial
conversion to hereditary base $2$ is not covered, but the material is
sufficient to compute the particular value $\mathcal{G}(16)$, and
easily extends to other fixed arguments.
\end{abstract}

\tableofcontents

\section{Introduction}

Given a number $n$ and a base $b$, we can write $n$ in
\emph{hereditary base $b$}, which results from writing $n$ in
base $b$, and then each exponent in hereditary base $b$ again.
For example, $7$ in hereditary base $3$ is $3^1 \cdot 2 + 1$.
Given the hereditary base $b$ representation of $n$,
we can reinterpret it in base $b+1$
by replacing all occurrences of $b$ by $b+1$.

The Goodstein sequence starting at $n$ in base $2$ is obtained by
iteratively taking a number in hereditary base $b$,
reinterpreting it in base $b+1$, and subtracting $1$.
The next step is the same with $b$ incremented by $1$, and so on.
So starting for example at $4$, we compute
\begin{align*}
4 = 2^{2^1}
&\:\to\: 3^{3^1} - 1
= 26\\
26 = 3^2 \cdot 2 + 3^1 \cdot 2 + 2
&\:\to\:
4^2 \cdot 2 + 4^1 \cdot 2 + 1 \cdot 2 - 1
= 41\\
41 =
4^2 \cdot 2 + 4^1 \cdot 2 + 1
&\:\to\:
5^2 \cdot 2 + 5^1 \cdot 2 + 1 - 1
= 60
\end{align*}
and so on.
We stop when we reach $0$.
Goodstein's theorem states that this process always terminates~\cite{G44}.
This result is independent of Peano Arithmetic,
and is intimately connected to countable ordinals and the
slow growing hierarchy (e.g., the Hardy function)~\cite{C83}.
The length of the resulting sequence is the Goodstein function,
denoted by $\mathcal G(n)$.
For example, $\mathcal G(3) = 6$.

For this formalization, we are interested in implementing the Goodstein
function in $\lambda$-calculus.
More concretely, we want to define the value $\mathcal G(16)$
(which is huge; for example, it exceeds Graham's number),
in order to bound its Kolmogorov complexity.
Our concrete measure of Kolmogorov complexity is the program length
in the Binary Lambda Calculus~\cite{BLC,T08}.
It turns out that we can define $\mathcal G(16)$ as follows,
giving a complexity bound of 195 bits.
\begin{align*}
\mathit{exp\omega} &=
(\lambda z\:s\:l.\:n\:s\:(\lambda x\:z.\:l\:(\lambda n.\:n\:x\:z))\:
(\lambda f\:z.\:l\:(\lambda n.\:f\:n\:z))\:z)\\
\mathit{goodstein} &= (\lambda n\:c.\:n\\
&\phantom{{}=(}(\lambda x.\:x)\\
&\phantom{{}=(}(\lambda n\:m.\:n\:(\lambda f\:x.\:m\:f\:(f\:x)))\\
&\phantom{{}=(}(\lambda f\:m.\:f\:(\lambda f\:x.\:m\:f\:(f\:(f\:x)))\:m)\\
&\phantom{{}=(}c)
\\
\mathcal G_{16} &=
(\lambda e.\:\mathit{goodstein}\:
(e\:(e\:(e\:(e\:(\lambda z\:s\:l.\:z)))))\:(\lambda x.\:x))\:
\mathit{exp\omega}
\end{align*}
We rely on a shallow embedding of the $\lambda$-calculus throughout
the formalization, so it turns out that we cannot quite prove this
claim in Isabelle/HOL; the expression for $\mathcal G_{16}$ cannot
be typed.
However, we can prove that the building blocks $\mathit{exp\omega}$
and $\mathit{goodstein}$ work correctly in the sense that
\begin{itemize}
\item $\mathit{exp\omega}^4\:(\lambda z\:s\:l.\:z)$
  is the hereditary base $2$ representation of $16$; and
\item $\mathit{goodstein}\:c\:n$ computes the length of a Goodstein
  sequence given that the hereditary base $c+1$ representation of
  the $c$-th value in the sequence is equal to $n$.
\end{itemize}
The remaining steps are easily verified by a human.

\paragraph{Contributions.}
Our main contributions are a concise specification of the Goodstein function,
another proof of Goodstein's theorem,
and establishing the connection to $\lambda$-calculus as already outlined.

\paragraph{Related work.}
There is already a formalization of Goodstein's theorem in the
AFP entry on nested multisets~\cite{NMO},
which comes with a formalization of ordinal arithmetic.
Our focus is different,
since our goal is to obtain an implementation of the Goodstein function
in $\lambda$-calculus.
Most notably, the intermediate type $\mathit{Ord}$ that we use to
represent ordinal numbers has far more structure than the ordinals themselves.
In particular it can represent arbitrary trees;
if we were to compute $\omega + 1$, $1 + \omega$ and $\omega$ on this type,
we would get three different results.
However, we will use the operations such that $1 + \omega$ is never computed,
keeping the connection to countable ordinals intact.
Proving this is a large, albeit hidden, part of our formalization.

\paragraph{Acknowledgement.}
John Tromp raised the question of a concise $\lambda$-calculus term computing
$\mathcal{G}(16)$.
He also provided feedback on a draft version of this document.

% sane default for proof documents
\parindent 0pt\parskip 0.5ex

% generated text of all theories
\input{session}

% optional bibliography
\bibliographystyle{abbrv}
\bibliography{root}

\end{document}

%%% Local Variables:
%%% mode: latex
%%% TeX-master: t
%%% End:
