\documentclass[11pt,a4paper]{book}
\usepackage{isabelle,isabellesym}
\usepackage{amssymb}
\usepackage[english]{babel}
\usepackage{caption} 
\usepackage[flushleft]{threeparttable}

% this should be the last package used
\usepackage{pdfsetup}

% urls in roman style, theory text in math-similar italics
\urlstyle{rm}
\isabellestyle{it}

\captionsetup[table]{skip=10pt}

\makeatletter
\newenvironment{abstract}{%
  \small
  \begin{center}%
    {\bfseries \abstractname\vspace{-.5em}\vspace{\z@}}%
  \end{center}%
  \quotation}{\endquotation}
\makeatother

% HACK: It's required to align multiline definitions and lemmas
\renewcommand{\isachardoublequoteopen}{\ }
\renewcommand{\isachardoublequoteclose}{\ }

\begin{document}

\title{Safe OCL}
\author{Denis Nikiforov}
\maketitle

\begin{abstract}
  The theory is a formalization of the OCL type system,
  its abstract syntax and expression typing rules~\cite{OCL24}.
  The theory does not define a concrete syntax and a semantics.
  In contrast to Featherweight OCL~\cite{Featherweight_OCL-AFP},
  it is based on a deep embedding approach. The type system is defined
  from scratch, it is not based on the Isabelle HOL type system.

  The Safe OCL distincts nullable and non-nullable types. Also
  the theory gives a formal definition of safe navigation
  operations~\cite{DBLP:conf/models/Willink15}. The Safe OCL typing rules
  are much stricter than rules given in the OCL specification.
  It allows one to catch more errors on a type checking phase.

  The type theory presented is four-layered: classes, basic types,
  generic types, errorable types. We introduce the following new types:
  non-nullable types (\isa{{\isasymtau}{\isacharbrackleft}{\isadigit{1}}{\isacharbrackright}}),
  nullable types (\isa{{\isasymtau}{\isacharbrackleft}{\isacharquery}{\isacharbrackright}}),
  \isa{OclSuper}. \isa{OclSuper} is a supertype of all other types
  (basic types, collections, tuples). This type allows us to define
  a total supremum function, so types form an upper semilattice.
  It allows us to define rich expression typing rules in an elegant manner.

  The Preliminaries Section of the theory defines a number of
  helper lemmas for transitive closures and tuples. It defines also
  a generic object model independent from OCL. It allows one to use
  the theory as a reference for formalization of analogous languages.
\end{abstract}

\tableofcontents

% include generated text of all theories
\input{session}

\bibliographystyle{ieeetr}
\bibliography{root}

\end{document}
