\documentclass[11pt,a4paper]{book}
\usepackage{isabelle,isabellesym}
\usepackage{amssymb}
\usepackage[english]{babel}
\usepackage[only,bigsqcap]{stmaryrd}

% this should be the last package used
\usepackage{pdfsetup}

% urls in roman style, theory text in math-similar italics
\urlstyle{rm}
\isabellestyle{it}

% Tweaks
\newcounter{TTStweak_tag}
\setcounter{TTStweak_tag}{0}
\newcommand{\setTTS}{\setcounter{TTStweak_tag}{1}}
\newcommand{\resetTTS}{\setcounter{TTStweak_tag}{0}}
\newcommand{\insertTTS}{\ifnum\value{TTStweak_tag}=1 \ \ \ \fi}

\renewcommand{\isakeyword}[1]{\resetTTS\emph{\bf\def\isachardot{.}\def\isacharunderscore{\isacharunderscorekeyword}\def\isacharbraceleft{\{}\def\isacharbraceright{\}}#1}}
\renewcommand{\isachardoublequoteopen}{\insertTTS}
\renewcommand{\isachardoublequoteclose}{\setTTS}
\renewcommand{\isanewline}{\mbox{}\par\mbox{}\resetTTS}


\renewcommand{\isamarkupcmt}[1]{\hangindent5ex{\isastylecmt --- #1}}

\makeatletter
\newenvironment{abstract}{%
  \small
  \begin{center}%
    {\bfseries \abstractname\vspace{-.5em}\vspace{\z@}}%
  \end{center}%
  \quotation}{\endquotation}
\makeatother

\begin{document}

\title{Refinement for Monadic Programs}
\author{Peter Lammich}
\maketitle

\begin{abstract}
  We provide a framework for program and data refinement in Isabelle/HOL.
  The framework is based on a nondeterminism-monad with assertions, i.e.,
  the monad carries a set of results or an assertion failure.
  Recursion is expressed by fixed points. For convenience, we also provide
  while and foreach combinators.

  The framework provides tools to automatize canonical tasks, such as 
  verification condition generation, finding appropriate data refinement relations,
  and refine an executable program to a form that is accepted by the 
  Isabelle/HOL code generator.

  Some basic usage examples can be found in this entry, but most of the examples and 
  the userguide have been moved to the Collections AFP entry. For more advanced examples,
  consider the AFP entries that are based on the Refinement Framework.
\end{abstract}

\clearpage
\tableofcontents

\clearpage

% sane default for proof documents
\parindent 0pt\parskip 0.5ex

\chapter{Introduction}
Isabelle/HOL\cite{NPW02} is a higher order logic theorem prover. Recently, we
started to use it to implement automata algorithms 
(e.g., \cite{L09_tree_automata}).
There, we do not only want to specify an algorithm and prove it correct, but
we also want to obtain efficient executable code from the formalization.
This can be done with Isabelle/HOL's code generator \cite{Haft09,HaNi10}, that 
converts
functional specifications inside Isabelle/HOL to executable programs. In order
to obtain a uniform interface to efficient data structures, we developed the
Isabelle Collection Framework (ICF) \cite{L09_collections,LL10}. 
It provides a uniform interface to various (collection) data structures, as well
as generic algorithm, that are parametrized over the data structure actually used,
and can be instantiated for any data structure providing the required operations.
E.g., a generic algorithm may be parametrized over a set data structure, and then
instantiated with a hashtable or a red-black tree.

The ICF features a data-refinement approach to prove an algorithm correct:
First, the algorithm is specified using the abstract data structures. These are
usually standard datatypes on Isabelle/HOL, and thus enjoy a good tool support
for proving. Hence, the correctness proof is most conveniently performed on this
abstract level. In a next step, the abstract algorithm is refined to a 
concrete algorithm that uses some efficient data structures. Finally, it is 
shown that the result of the concrete algorithm is related to the result of 
the abstract algorithm. This last step is usually fairly straightforward.

This approach works well for simple operations. However, it is not 
applicable when using inherently nondeterministic operations on the abstract level,
such as choosing an arbitrary element from a non-empty set. In this case, any 
choice of the element on the abstract level
over-specifies the algorithm, as it forces the concrete algorithm to choose the
same element.

One possibility is to initially specify and prove correct the algorithm on 
the concrete level, possibly using parametrization to leave the concrete
implementation unspecified. The problem here is, that the correctness proofs
have to be performed on the concrete level, involving abstraction steps during 
the proof, which makes it less readable and more tedious. Moreover, this 
approach does not support stepwise refinement, as all operations have to work on
the most concrete datatypes.

Another possibility is to use a non-deterministic algorithm on the abstract 
level, that is then refined to a deterministic algorithm. Here, the 
correctness proofs may be done on the abstract level, and stepwise refinement is
properly supported.

However, as Isabelle/HOL  primarily supports functions, not relations,
formulating   nondeterministic  algorithms   is  more   tedious.  This
development  provides  a  framework for  formulating  nondeterministic
algorithms in a  monadic style, and using program  and data refinement
to  eventually obtain an  executable algorithm.  The monad  is defined
over a set of results and a special {\em FAIL}-value, that indicates a
failed assertion.  The framework provides some tools to make reasoning
about those monadic programs more comfortable.

\section{Related Work}
Data refinement dates back to Hoare \cite{Hoa72}. Using 
{\em refinement calculus} for stepwise program refinement, including data
refinement, was first proposed by Back \cite{Back78}.  In the last
decades, these topics have been subject to extensive research. Good
overviews are \cite{BaWr98,RoEn98}, that cover the main concepts on
which this formalization is based.  There are various formalizations
of refinement calculus within theorem provers
\cite{BaWr90,LRW95,RuWr97,Stap99,Preo06}.  All these works focus on
imperative programs and therefore have to deal with the representation
of the state space (e.g., local variables, procedure parameters).  In
our monadic approach, there is no need to formalize state spaces or
procedures, which makes it quite simple. Note, that we achieve
modularization by defining constants (or recursive functions), thus
moving the burden of handling parameters and procedure calls to the
underlying theorem prover, and at the same time achieving a more
seamless integration of our framework into the theorem prover. 
In the seL4-project \cite{CKS08}, a nondeterministic state-exception monad is
used to refine the abstract specification of the kernel to an
executable model. The basic concept is closely related to ours.
However, as the focus is different (Verification of kernel operations vs. 
verification of model-checking algorithms), there are some major differences in
the handling of recursion and data refinement. 
In \cite{SchM98}, {\em refinement monads} are studied. 
The basic constructions there are similar to ours. However, while we focus on
data refinement, they focus on introducing commands with side-effects and a 
predicate-transformer semantics to allow angelic nondeterminism.

% generated text of all theories
\input{session}

\chapter{Conclusion and Future Work}
We have presented a framework for program and data refinement.
The notion of a program is based on a nondeterminism monad, and we provided
tools for verification condition generation, finding data 
refinement relations, and for generating executable code by 
Isabelle/HOL's code generator \cite{Haft09,HaNi10}.

We illustrated the usability of our framework by various examples, among others
a breadth-first search algorithm, which was our solution to task~5 of the VSTTE 
2012 verification competition.

There is lots of possible future work. We sketch some major directions here:
\begin{itemize}
\item
Some of our refinement rules (e.g.\ for while-loops) are only applicable for
single-valued relations. This seems to be related to the monadic structure of
our programs, which focuses on single values. A direction of future research is
to understand this connection better, and to develop usable rules for 
non single-valued abstraction relations.
\item
Currently, transfer for partial correct programs is done to a complete-lattice
domain. However, as assertions need not to be included in the transferred program,
we could also transfer to a ccpo-domain, as, e.g., the option monad that is 
integrated into Isabelle/HOL by default. This is, however, only a technical 
problem, as ccpo and lattice typeclasses are not properly 
linked\footnote{This has also been fixed in the development version of 
Isabelle/HOL}.
Moreover, with the partial function package \cite{Kr10}, 
Isabelle/HOL has a powerful tool to express arbitrary recursion schemes over 
monadic programs. Currently, we have done the basic setup for the partial 
function package, i.e., we can define recursions over our monad. However, 
induction-rule generation does not yet work, and there is potential for more 
tool-support regarding refinement and transfer to deterministic programs.
\item
Finally, our framework only supports functional programs. However, as shown in
Imperative/HOL \cite{BKHEM08}, monadic programs are well-suited to 
express a heap. Hence, a direction of future research is to add a heap to our
nondeterminism monad. Argumentation about the heap could be done with a 
separation logic \cite{Rey02} formalism, like the one that we already 
developed for Imperative/HOL \cite{Meis2011}.
\end{itemize}

% optional bibliography
\bibliographystyle{abbrv}
\bibliography{root}

\end{document}

%%% Local Variables:
%%% mode: latex
%%% TeX-master: t
%%% End:
