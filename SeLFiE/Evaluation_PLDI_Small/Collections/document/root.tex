\documentclass[11pt,a4paper]{book}
\usepackage{isabelle,isabellesym}
\usepackage{amssymb}
\usepackage[english]{babel}
\usepackage[only,bigsqcap]{stmaryrd}

% this should be the last package used
\usepackage{pdfsetup}

% urls in roman style, theory text in math-similar italics
\urlstyle{rm}
\isabellestyle{it}

% Tweaks
\newcounter{TTStweak_tag}
\setcounter{TTStweak_tag}{0}
\newcommand{\setTTS}{\setcounter{TTStweak_tag}{1}}
\newcommand{\resetTTS}{\setcounter{TTStweak_tag}{0}}
\newcommand{\insertTTS}{\ifnum\value{TTStweak_tag}=1 \ \ \ \fi}

\renewcommand{\isakeyword}[1]{\resetTTS\emph{\bf\def\isachardot{.}\def\isacharunderscore{\isacharunderscorekeyword}\def\isacharbraceleft{\{}\def\isacharbraceright{\}}#1}}
\renewcommand{\isachardoublequoteopen}{\insertTTS}
\renewcommand{\isachardoublequoteclose}{\setTTS}
\renewcommand{\isanewline}{\mbox{}\par\mbox{}\resetTTS}

\renewcommand{\isamarkupcmt}[1]{\hangindent5ex{\isastylecmt --- #1}}

\newcommand{\isaheader}[1]{#1}
\newcommand{\isachapter}[1]{\chapter{#1}}
\newcommand{\isasection}[1]{\section{#1}}

\renewcommand{\isamarkupchapter}[1]{\chapter{#1}}
\renewcommand{\isamarkupsection}[1]{\subsection{#1}}
\renewcommand{\isamarkupsubsection}[1]{\subsubsection{#1}}
\renewcommand{\isamarkupsubsubsection}[1]{\paragraph{#1}}

\makeatletter
\newenvironment{abstract}{%
  \small
  \begin{center}%
    {\bfseries \abstractname\vspace{-.5em}\vspace{\z@}}%
  \end{center}%
  \quotation}{\endquotation}
\makeatother

% Macros for interface and implementation documentation

\newcommand{\docIntf}[2]{

{\bf Interface} #1 (theory #2)\\}

\newcommand{\docAbstype}[1]{\hspace*{2em}Abstract type: #1\\}

\newcommand{\docDesc}[1]{#1}

\newenvironment{docImpls}{

{\bf Implementations:}
\begin{description}%
}{\end{description}}

\newcommand{\docImpl}[1]{\item[#1]}

\newcommand{\docType}[1]{(Type: #1)}

\newcommand{\docAbbrv}[1]{(Abbrv: #1)}


\begin{document}

\title{Isabelle Collections Framework}
\author{By Peter Lammich and Andreas Lochbihler}
\maketitle

\begin{abstract}
  This development provides an efficient, extensible, machine checked collections framework for use
  in Isabelle/HOL. The library adopts the concepts of interface, implementation and generic algorithm from
  object-oriented programming and implements them in Isabelle/HOL.

  The framework features the use of data refinement techniques to refine an abstract specification (using high-level concepts like sets) to a more concrete implementation (using collection datastructures, like red-black-trees). The code-generator of Isabelle/HOL can be used to generate efficient code in all supported target languages, i.e. Haskell, SML, and OCaml.
\end{abstract}

\clearpage

\tableofcontents

\clearpage

% sane default for proof documents
\parindent 0pt\parskip 0.5ex

\chapter{Prologue}
\label{cha:pro}


\begin{quote}
  Verifying more examples of probabilistic algorithms will inevitably
  necessitate more formalization; in particular we already can see
  that a theory of expectation will be required to prove the
  correctness of probabilistic quicksort. If we can continue our
  policy of formalizing standard theorems of mathematics to aid
  verifications, then this will provide long-term benefits to many
  users of the HOL theorem prover.      
\end{quote}

This quote from the Future Work section of Joe Hurd's PhD thesis
``Formal Verification of Probabilistic Algorithms'' (\cite{hurd2002}
p. 131) 
served as a starting point for the following work. A theory of
expectation is nothing but a theory of integration in its probability
theoretic underpinnings. And though the proof of correctness for
probabilistic quicksort might not need integration, an average runtime
analysis certainly will.  

As indicated in the very beginning, integration is needed in some way
to talk about expectation in probability. The notion that is addressed
here is a kind of average value of a random variable with respect to a
(probability) measure. The concept of a \textit{measure} lies at the
heart of Lebesgue integration. A measure is simply a function
satisfying a few sanity properties that maps sets to real numbers.
Because the definition does not employ such concrete entities as
intervals, it generalizes easily to functions that do not have the
real numbers as their domain. In particular, the notion of measure is
very natural in the field of probability theory, where a probability
measure --- nothing but a measure $P$ with $P(\Omega)=1$ --- gives the
probability of an event --- a measurable subset of $\Omega$.

This $\Omega$ might, for example, be the set of all infinite sequences
of boolean values, as in Hurd's thesis\cite{hurd2002}; our integral is
then just a tool that extends this work in the sense depicted at the
very beginning of this introduction. 

We begin by declaring some preliminary notions, including
elementary measure theory and monotone convergence. This leads into
measurable real-valued functions, also known as random variables. A
sufficient body of functions is shown to belong to this class. 
The central chapter is about integration proper. We build the integral
for increasingly complex functions and prove essential properties,
discovering the connection with measurability in the end. 

 
\chapter{Measurable Functions}
\label{cha:real-valued-random}

In this chapter, the focus is on the kind of functions to be 
integrated. As we will see later on, measurability is a
good characterization for these functions. Moreover, the language of
measure theory as well as the notion of monotone convergence is used
frequently in the definition of the integral. So we begin by formalizing
these necessary tools.

\section{Preliminaries}
\label{sec:preliminaries}


% LocalWords:  quicksort

%%% Local Variables: 
%%% mode: latex
%%% TeX-master: "root"
%%% End: 

% generated text of all theories
\input{session}

\section{Conclusion}\label{sec:concl}
  We have presented a verification of two variants of Gabow's algorithm: Computation of the strongly connected components of
  a graph, and emptiness check of a generalized B\"uchi automaton. We have extracted efficient code with a performance comparable to a
  reference implementation in Java.
  
  We have modularized the formalization in two directions: First, we share most of the proofs between the two variants of the algorithm. Second,
  we use a stepwise refinement approach to separate the algorithmic ideas and the correctness proof from implementation details.
  Sharing of the proofs reduced the overall effort of developing both algorithms. Using a stepwise refinement approach allowed us to
  formalize an efficient implementation, without making the correctness proof complex and unmanageable by cluttering it with implementation details.

  Our development approach is independent of Gabow's algorithm, and can be re-used for the verification of other algorithms.

  \paragraph{Current and Future Work} 
  An important direction of future work is to fine-tune the implementation of 
  the emptiness check algorithm for speed, as speed of the checking algorithm
  directely influences the performance of the modelchecker.


% optional bibliography
\bibliographystyle{abbrv}
\bibliography{root}

\end{document}

%%% Local Variables:
%%% mode: latex
%%% TeX-master: t
%%% End:
