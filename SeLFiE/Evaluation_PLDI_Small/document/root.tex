\documentclass[11pt,a4paper]{article}
\usepackage{isabelle,isabellesym}
\usepackage{amsmath}
\usepackage[english]{babel}
\usepackage{stmaryrd}
\usepackage{eufrak}
\usepackage{wasysym}
\usepackage{tikz}

% this should be the last package used
\usepackage{pdfsetup}

% urls in roman style, theory text in math-similar italics
\urlstyle{rm}
\isabellestyle{it}


\begin{document}

\title{Markov Models}
\author{Johannes H\"olzl and Tobias Nipkow}
\maketitle

\begin{abstract}
This is a formalization of various Markov models in Isabelle/HOL. It builds on Isabelle's
probability theory. The available models are currently discrete-time and continuous-time Markov
chains as well as Markov decision processes. As application of these models we formalize
probabilistic model checking of pCTL formulas, analysis of IPv4 address allocation in ZeroConf and
an analysis of the anonymity of the Crowds protocol.
\end{abstract}

\tableofcontents

\section{Introduction}

This is a formalization of probabilistic models in Isabelle/HOL. It builds on Isabelle's probability
theory (HOL-Probability). It provides formalizations for the following models:

\begin{itemize}
\item Discrete-time Markov processes with measurable state spaces~\cite{hoelzl2017markovprocesses}
\item Markov decision processes on discrete spaces~\cite{hoelzl2017mdp}
\item Continuous-time Markov chains on discrete spaces~\cite{hoelzl2017markovprocesses}
\end{itemize}

As application of these models we formalize
\begin{itemize}
\item a probabilistic model checking of pCTL formulas~\cite{hoelzl2012verifyingpctl},
\item an analysis of IPv4 address allocation in ZeroConf~\cite{hoelzl2012casestudies},
\item an analysis of the anonymity of the Crowds protocol~\cite{hoelzl2012casestudies},
\item the reachability analysis on finite-state MDPs~\cite{hoelzl2017mdp}, and
\item expected running-time semantics for pGCL~\cite{hoelzl2016exprun}.
\end{itemize}

The formalization of rewarded DTMCs and pCTL model checking is discussed in
detail in our paper.

% sane default for proof documents
\parindent 0pt\parskip 0.5ex

% generated text of all theories
\input{session}

% optional bibliography
\bibliographystyle{abbrv}
\bibliography{root}

\end{document}

%%% Local Variables:
%%% mode: latex
%%% TeX-master: t
%%% End:
