\documentclass[11pt,a4paper]{article}
\usepackage[utf8]{inputenc}
\usepackage[margin=2cm]{geometry}
\usepackage{isabelle,isabellesym}
\usepackage{amsmath}
\usepackage{amssymb}
\usepackage[english]{babel}
\usepackage{stmaryrd}
\usepackage{eufrak}
\usepackage{wasysym}
\usepackage{tikz}

% this should be the last package used
\usepackage{pdfsetup}

% urls in roman style, theory text in math-similar italics
\urlstyle{rm}
\isabellestyle{it}


\begin{document}


\title{Probabilistic Timed Automata}
\author{Simon Wimmer and Johannes Hölzl}
\maketitle

\begin{abstract}
We present a formalization of probabilistic timed automata (PTA) for which we try to follow the formula
``MDP + TA = PTA'' as far as possible:
our work starts from our existing formalizations of Markov decision processes (MDP)
and timed automata (TA) and combines them modularly.
We prove the fundamental result for probabilistic timed automata:
the region construction that is known from
timed automata carries over to the probabilistic setting.
In particular, this allows us to prove that minimum and maximum reachability probabilities
can be computed via a reduction to MDP model checking,
including the case where one wants to disregard unrealizable behavior.
Further information can be found in our ITP paper \cite{PTA-ITP-2018}.

\end{abstract}

The definition of the PTA semantics can be found in Section~\ref{sem:mdp},
the region MDP is in Section~\ref{sem:mdprg},
the bisimulation theorem is in Section~\ref{thm:bisim},
and the final theorems can be found in Section~\ref{thm:minmax}.
The background theory we formalize is described in the seminal paper on
PTA \cite{KNSS2002}.

\tableofcontents

\pagebreak

% sane default for proof documents
\parindent 0pt\parskip 0.5ex

% generated text of all theories
\input{session}

% optional bibliography
\bibliographystyle{abbrv}
\bibliography{root}

\end{document}

%%% Local Variables:
%%% mode: latex
%%% TeX-master: t
%%% End:
