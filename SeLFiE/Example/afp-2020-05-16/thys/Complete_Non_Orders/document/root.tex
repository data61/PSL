\documentclass[11pt,a4paper]{article}
\usepackage{isabelle,isabellesym}
\usepackage{amssymb,amsmath,stmaryrd}
\usepackage{tikz}
\usetikzlibrary{backgrounds}
\usetikzlibrary{positioning}
\usetikzlibrary{shapes}

% this should be the last package used
\usepackage{pdfsetup}

% urls in roman style, theory text in math-similar italics
\urlstyle{rm}
\isabellestyle{it}

\newcommand\SLE{\sqsubseteq}
\newcommand\Nat{\mathbb{N}}


\makeatletter

\def\tp@#1#2{\@ifnextchar[{\tp@@{#1}{#2}}{\tp@@@{#1}{#2}}}
\def\tp@@#1#2[#3]#4{#3#1\def\mid{\mathrel{#3|}}#4#3#2}
\def\tp@@@#1#2#3{\bgroup\left#1\def\mid{\;\middle|\;}#3\right#2\egroup}
\def\pa{\tp@()}
\def\tp{\tp@\langle\rangle}
\def\set{\tp@\{\}}

\makeatother


\begin{document}

\title{Complete Non-Orders and Fixed Points}
\author{Akihisa Yamada and Jérémy Dubut}
\maketitle

\begin{abstract}
  We develop an Isabelle/HOL library of order-theoretic concepts, such as various completeness conditions and
   fixed-point theorems. We keep our formalization as general as possible: we reprove several well-known results 
   about complete orders, often without any properties of ordering, thus complete non-orders. In particular, we 
   generalize the Knaster--Tarski theorem so that we ensure the existence of a quasi-fixed point of monotone maps 
   over complete non-orders, and show that the set of quasi-fixed points is complete under a mild condition---%
   attractivity---which is implied by either antisymmetry or transitivity. This result generalizes and strengthens a result
    by Stauti and Maaden. Finally, we recover Kleene's fixed-point theorem for omega-complete non-orders, again 
    using attractivity to prove that Kleene's fixed points are least quasi-fixed points.
\end{abstract}

\tableofcontents

\section{Introduction}

The main driving force towards mechanizing mathematics using proof assistants
has been the reliability they offer,
exemplified prominently by~\cite{4color},~\cite{flyspeck},~\cite{sel4}, etc.
In this work, we utilize another aspect of Isabelle/JEdit~\cite{isabelle/jedit}
as engineering tools for developing mathematical theories.
We formalize order-theoretic concepts and results,
adhering to an \emph{as-general-as-possible} approach:
most results concerning order-theoretic completeness and fixed-point theorems
are proved without assuming the underlying relations to be orders (non-orders).
In particular, we provide the following:
\begin{itemize}
\item
A locale-based library for binary relations,
as partly depicted in Figure~\ref{fig:non-orders}.
\item
Various completeness results that generalize known theorems in order theory:
Actually most relationships and duality of completeness conditions are proved without
\emph{any} properties of the underlying relations.
\item Existence of fixed points:
We show that a relation-preserving mapping $f : A \to A$
over a complete non-order $\tp{A,\SLE}$
admits a \emph{quasi-fixed point} $f(x) \sim x$,
meaning $x \SLE f(x) \wedge f(x) \SLE x$.
Clearly if $\SLE$ is antisymmetric then this implies the existence of fixed points $f(x) = x$.
\item Completeness of the set of fixed points:
We further show that
if $\SLE$ satisfies a mild condition, which we call \emph{attractivity} and
which is implied by either transitivity or antisymmetry,
then the set of quasi-fixed points is complete.
Furthermore, we also show that if $\SLE$ is antisymmetric,
then the set of \emph{strict} fixed points $f(x) = x$ is complete.
\item Kleene-style fixed-point theorems:
For an $\omega$-complete non-order $\tp{A,\SLE}$ with a bottom element $\bot \in A$ (not necessarily unique)
and for every $\omega$-continuous map $f : A \to A$,
a supremum exists for the set $\set{ f^n(\bot) \mid n \in \Nat}$,
and it is a quasi-fixed point.
If $\SLE$ is attractive, then
the quasi-fixed points obtained this way are precisely the least quasi-fixed points.
\end{itemize}
We remark that all these results would have required much more effort than we spent
(if possible at all),
if we were not with the smart assistance by Isabelle.
Our workflow was often the following: first we formalize existing proofs, try relaxing assumptions, see where proof breaks,
and at some point ask for a counterexample.

\begin{figure}
\small
\centering
\def\isa#1{\textsf{#1}}
\def\t{-1.8}
\def\a{3.6}
\def\at{1.8}
\def\s{-3.6}
\def\st{-5.4}
\begin{tikzpicture}
\tikzstyle{every node}=[draw,ellipse]
\tikzstyle{every edge}=[draw]
\draw
 (0,0) node[fill] (rel) {}
 (\t,1) node (trans) {\isa{transitive}}
 (0,-2) node (refl) {\isa{reflexive}}
 (0,2) node (irr) {\isa{irreflexive}}
 (\s,0) node (sym) {\isa{symmetric}}
 (\a,0) node (anti) {\isa{antisymmetric}}
 (\at,1) node (near) {\isa{near\_order}}
 (\a,2) node (asym) {\isa{asymmetric}}
 (\a,-2) node (pso) {\isa{pseudo\_order}}
 (\at,-1) node (po) {\isa{partial\_order}}
 (\t,-1) node (qo) {\isa{quasi\_order}}
 (0,3) node (str) {\hspace{3em}\isa{strict\_order}\mbox{\hspace{3em}}}
 (\st,-1) node (equiv) {\isa{equivalence}}
 (\st,1) node (peq) {\hspace{-.8em}\isa{partial\_equivalence}\mbox{\hspace{-.8em}}}
 (\st,3) node (emp) {$\emptyset$}
 (\s,-2) node (tol) {\isa{tolerance}}
 (\s,2) node (ntol) {$\neg$\isa{tolerance}}
 ;
\draw[->]
 (near) edge[color=blue] ([xshift=51,yshift=-7]str)
 (irr) edge[color=red] ([xshift=-46,yshift=-8]str)
 (trans) edge[color=blue] ([xshift=-51,yshift=-7]str)
 (asym) edge[color=red] ([xshift=55,yshift=-7]str)
 (trans) edge[color=green] (near)
 (anti) edge[color=red] (near)
 (irr) edge[color=green] (asym)
 (anti) edge[color=blue] (asym)
 (anti) edge[color=blue] (pso)
 (near) edge[color=blue] (po)
 (pso) edge[color=red] (po)
 (refl) edge[color=green] (pso)
 (trans) edge[color=blue] (qo)
 (refl) edge[color=red] (qo)
 (qo) edge[color=green] (po)
 (qo) edge[color=green] (equiv)
 (peq) edge[color=blue] (equiv)
 (trans) edge[color=green] (peq)
 (peq) edge[color=blue] (emp)
 (str) edge[color=green] (emp)
 (sym) edge[color=red] (peq)
 (sym) edge[color=blue] (tol)
 (sym) edge[color=blue] (ntol)
 (irr) edge[color=green] (ntol)
 (refl) edge[color=green] (tol)
 (ntol) edge[color=red] (emp)
 (tol) edge[color=red] (equiv)
 (rel) edge[color=red, line width=1.5pt] (trans)
 (rel) edge[color=blue, line width=1.5pt] (irr)
 (rel) edge[color=blue, line width=1.5pt] (refl)
 (rel) edge[color=green, line width=1.5pt] (sym)
 (rel) edge[color=green, line width=1.5pt] (anti)
 ;
\end{tikzpicture}
\caption{\label{fig:non-orders}
Combinations of basic properties.
The black dot around the center represents arbitrary binary relations,
and the five outgoing arrows indicate atomic assumptions.
We do not present the combination
of \isa{reflexive} and \isa{irreflexive}, which is empty,
and one of \isa{symmetric} and \isa{antisymmetric},
which is a subset of equality.
Node ``$\neg$\isa{tolerance}'' indicates the negated relation is \isa{tolerance},
and ``$\emptyset$'' is the empty relation.
}
\end{figure}

\paragraph*{Related Work}
Many attempts have been made to generalize the notion of completeness for lattices, conducted in different directions: by relaxing the notion of order itself, removing transitivity (pseudo-orders \cite{trellis}); by relaxing the notion of lattice, considering minimal upper bounds instead of least upper bounds ($\chi$-posets \cite{LN83}); by relaxing the notion of completeness, requiring the existence of least upper bounds for restricted classes of subsets (e.g., directed complete and $\omega$-complete, see \cite{davey02} for a textbook). Considering those generalizations, it was natural to prove new versions of classical fixed-point theorems for maps preserving those structures, e.g., existence of least fixed points for monotone maps on (weak chain) complete pseudo-orders \cite{Bhatta05, SM13}, construction of least fixed points for $\omega$-continuous functions for $\omega$-complete lattices \cite{mashburn83}, (weak chain) completeness of the set of fixed points for monotone functions on (weak chain) complete pseudo-orders \cite{PG11}.

Concerning Isabelle formalization,
one can easily find several formalizations of complete partial orders or lattices in Isabelle's standard library.
They are, however, defined on partial orders, either in form of classes or locales,
and thus not directly reusable for non-orders.
Nevertheless we tried to make our formalization compatible with the existing ones,
and various correspondences are ensured.

This work has been published in the conference paper \cite{YamadaD2019}.

% include generated text of all theories
\input{session}

\bibliographystyle{abbrv}
\bibliography{root}

\end{document}
