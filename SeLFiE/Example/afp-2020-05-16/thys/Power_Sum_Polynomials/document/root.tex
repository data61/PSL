\documentclass[11pt,a4paper]{article}
\usepackage{isabelle,isabellesym}
\usepackage{amsfonts, amsmath, amssymb}

% this should be the last package used
\usepackage{pdfsetup}

% urls in roman style, theory text in math-similar italics
\urlstyle{rm}
\isabellestyle{it}

\begin{document}

\title{Power Sum Polynomials\\ and the Girard--Newton Theorem}
\author{Manuel Eberl}
\maketitle

\begin{abstract}
This article provides a formalisation of the symmetric multivariate polynomials known as \emph{power sum polynomials}. These are of the form $p_n(X_1,\ldots, X_k) = X_1 ^ n + \ldots + X_k ^ n$. A formal proof of the Girard--Newton Theorem is also given. This theorem relates the power sum polynomials to the elementary symmetric polynomials $s_k$ in the form of a recurrence relation $(-1)^k k s_k = \sum_{i=0}^{k-1} (-1)^i s_i p_{k-i}$\ .

As an application, this is then used to solve a generalised form of a puzzle given as an exercise in Dummit and Foote's \emph{Abstract Algebra}: For $k$ complex unknowns $x_1, \ldots, x_k$, define $p_j := x_1^j + \ldots + x_k^j$. Then for each vector $a\in\mathbb{C}^k$, show that there is exactly one solution to the system $p_1 = a_1, \ldots, p_k = a_k$ up to permutation of the $x_i$ and determine the value of $p_i$ for $i>k$.
\end{abstract}

\tableofcontents
\newpage
\parindent 0pt\parskip 0.5ex

\input{session}

\bibliographystyle{abbrv}
\bibliography{root}

\end{document}

%%% Local Variables:
%%% mode: latex
%%% TeX-master: t
%%% End:
