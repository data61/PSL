%  Title:       An Example of a Cofinitary Group in Isabelle/HOL
%
%  Author:      Bart.Kastermans at colorado.edu, 2009
%  Maintainer:  Bart.Kastermans at colorado.edu

\documentclass[11pt,a4paper]{article}
\usepackage{isabelle,isabellesym}
\usepackage {amsmath}

\usepackage{amssymb}

\def\polhk#1{\setbox0=\hbox{#1}{\ooalign{\hidewidth
    \lower1.5ex\hbox{`}\hidewidth\crcr\unhbox0}}} 

% this should be the last package used
\usepackage{pdfsetup}

% urls in roman style, theory text in math-similar italics
\urlstyle{rm}
\isabellestyle{it}

\DeclareMathOperator{\Sym}{Sym}
\newcommand {\N} {\ensuremath {\mathbb {N}}}

\begin{document}

\title{An Example of a Cofinitary Group in Isabelle/HOL}
\author{Bart Kastermans}

\maketitle

\begin{abstract}
  We formalize the usual proof that the group generated by the
  function $k \mapsto k + 1$ on the integers gives rise to a
  cofinitary group.
\end{abstract}

\tableofcontents

\vspace {.3cm}

% sane default for proof documents
\parindent 0pt\parskip 0.5ex

% generated text of all theories
\input{session}

\bibliographystyle{abbrv}
\bibliography{root}

\end{document}

%%% Local Variables:
%%% mode: latex
%%% TeX-master: t
%%% End:
