\documentclass[11pt,a4paper]{article}
\usepackage{listings,url}
%\DeclareUrlCommand\myURL{\def\UrlFont{\rmfamily}}
\urlstyle{rm}
\lstset{language=java,showstringspaces=false, mathescape=true,
morekeywords={member,export,replicating,module,own,as,with},
flexiblecolumns=false, basicstyle=\footnotesize\ttfamily, stringstyle=\itshape,
tabsize=2, escapechar=£, frame=lines}
\newcommand{\ie}{i.e.~}
\newcommand{\eg}{e.g.~}

%%\usepackage{isabelle,isabellesym}
%%\usepackage{pdfsetup}
%%\isabellestyle{it}

\begin{document}

\title{Lightweight Java}
\author{Rok Strni\v sa \and Matthew Parkinson}
\maketitle

\begin{abstract}
Lightweight Java (LJ) is an imperative fragment of Java~\cite{java}. It is
intended to be as simple as possible while still retaining the feel of Java. LJ
includes fields, methods, single inheritance, dynamic method dispatch, and
method overriding. It does not include support for local variables, field
hiding, interfaces, inner classes, or generics. The accompanying Isabelle
script proves the type soundness of the Ott-generated LJ definition.
\end{abstract}

\section{Description}

When designing or reasoning about a language feature or a language analysis,
researchers try to limit the underlying language to avoid dealing with
unnecessary details. For example, object-oriented generics were formalised on
top of Featherweight Java (FJ)~\cite{fj}, a substantially simplified model of
the Java programming language~\cite{java}.

Many researchers have used FJ as their base language. However, FJ is not always
suitable, since it is purely functional --- it does not model state; there are
only expressions, which are evaluated completely locally. Therefore, FJ is a
poor choice for language analyses or language features that rely on state, \eg
separation logic~\cite{sl} or mixins~\cite{mixins}.

In this chapter, we present Lightweight Java (LJ), a minimal {\em imperative}
core of Java. We chose a minimal set of features that still gives a Java-like
feel to the language, \ie fields, methods, single inheritance, dynamic method
dispatch, and method overriding. We did not include type casts, local
variables, field hiding, interfaces, method overloading, or any of the more
advanced language features mainly due to their apparent orthogonality to the
Java Module System~\cite{jsr277}, a research topic at the time; however, we
later realised that, by including type casts and static data, we could formally
verify properties regarding class cast exceptions (or their lack of) and module
state independence --- this extension remains future work.

LJ's semantics uses a program heap, and a variable state, but does not model a
frame stack --- method calls are effectively flattened as they are executed,
which simplifies the semantics. In spite of this, LJ is a proper subset of
Java, \ie every LJ program is a valid Java program, while its observable
semantics exactly corresponds to Java's semantics.

LJ is largely a simplification of Middleweight Java (MJ)~\cite{mj-matt}. In
addition to the above, MJ models a stack, type casts, and supports expressions
(not just statements).

LJ is defined rigorously. It is designed in Ott~\cite{ott}, a tool for writing
definitions of programming languages and calculi. From LJ's Ott code, the tool
also generates the language definition in Isabelle/HOL~\cite{isabelle}, a tool
for writing computer-verified maths. Based on this definition, we mechanically
prove type soundness in Isabelle/HOL, which gives us high confidence in the
correctness of the results.

Initially, we designed LJ as a base language for modelling the Java Module
System, Lightweight Java Module System (LJAM)~\cite{ljam}, and its improvement,
Improved Java Module System (iJAM)~\cite{iJAM} --- in both, we achieved a high
level of reuse in both the definitions and proof scripts.  Through this
process, LJ has been abstracted to the point where we think it can be used for
experimenting with other language features. In fact, LJ has already been used
by others to formalise ``features'' in Lightweight Feature Java~\cite{lfj}.

\section{Example program}

Here are two Lightweight Java class definitions, which show the use of class fields,
class methods, class inheritance, method overriding, subtyping, and dynamic
method dispatch.

\begin{lstlisting}
class A {                // class definition
    A f;                 // class field
    A m(B var) { this.f = var; return var; }    // subtyping
}

class B extends A {      // class inheritance
    A m(B var) { this.f = var; return this; }  // overriding
}

// A a, result; B b;
a = new B();             // subtyping
b = new B();
result = a.m(b);         // dynamic method dispatch (calls B::m)
\end{lstlisting}
\lstset{language=caml,morekeywords={match,with,then}}

Due to method overriding, the method call on the last line calls {\tt B}'s
method {\tt m}. Therefore, when the execution stops,
both {\tt result} and {\tt a} point to the same heap location.

\section{Extending the language}

The easiest way to extend the language is to modify its Ott source files. To
prove progress and well-formedness preservation of the extension, you can
either:

\begin{itemize}
\item modify the existing Isabelle scripts; or,
\item prove that any valid program of the extended language can be reduced to a
program in LJ.
\end{itemize}

\section{More information}

More information about Lightweight Java's operational semantics, type system,
type checking, and a detailed walkthrough of the proof of type soundness can be
found here:

\begin{center} \url{http://rok.strnisa.com/lj/} \end{center}

% include generated text of all theories
% \input{session}

\bibliographystyle{abbrv}
\bibliography{root}

\end{document}
