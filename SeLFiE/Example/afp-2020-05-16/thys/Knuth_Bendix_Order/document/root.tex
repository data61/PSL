\documentclass[11pt,a4paper]{article}
\usepackage{isabelle,isabellesym}

\usepackage{amssymb}
\usepackage{xspace}

% this should be the last package used
\usepackage{pdfsetup}

% urls in roman style, theory text in math-similar italics
\urlstyle{rm}
\isabellestyle{it}

\newcommand\isafor{\textsf{Isa\kern-0.15exF\kern-0.15exo\kern-0.15exR}}
\newcommand\ceta{\textsf{C\kern-0.15exe\kern-0.45exT\kern-0.45exA}}

\begin{document}

\title{A Formalization of Knuth--Bendix Orders\footnote{Supported by FWF (Austrian Science Fund) projects P27502 and Y757.}}
\author{Christian Sternagel and Ren\'e Thiemann}
\maketitle

\begin{abstract}
We define a generalized version of Knuth--Bendix orders, including
subterm coefficient functions. For these orders we formalize several
properties such as strong normalization, the subterm property, closure properties
under substitutions and contexts, as well as ground totality.
\end{abstract}

\tableofcontents

\section{Introduction}

In their seminal paper \cite{KB70}, Knuth and Bendix introduced two important
concepts: a procedure that allows us to solve certain instances of the word
problem -- (Knuth--Bendix) completion -- as well as a specific
order on terms that is useful to orient equations in the aforementioned
procedure -- the Knuth--Bendix order (or KBO, for short).

This AFP-entry is about the formalization of KBO.
Note that there are several variants of KBO~\cite{KB70,DKM90,LW07,ZHM09,S89}, e.g., incorporating quasi-precedences,
infinite signatures, subterm coefficient functions, and generalized 
weight functions. In fact, not for all of these variants well-foundedness has been
proven.
We give
the first well-foundedness proof for a variant of KBO that combines 
infinite signatures, quasi-precedences, and subterm coefficient functions.
Our proof is direct, i.e., it does not depend on Kruskal's tree
theorem.

This formalization is used in the \isafor/\ceta project~\cite{TS09b} for certifying untrusted
termination and confluence proofs. For more details we refer to our RTA paper \cite{paper}.

\input{session}



\bibliographystyle{abbrv}
\bibliography{root}

\end{document}

