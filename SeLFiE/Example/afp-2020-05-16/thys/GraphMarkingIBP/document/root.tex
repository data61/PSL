\documentclass[11pt,a4paper]{article}
\usepackage{isabelle,isabellesym}
\usepackage{amssymb}
\usepackage[only,bigsqcap]{stmaryrd}

% this should be the last package used
\usepackage{pdfsetup}

% urls in roman style, theory text in math-similar italics
\urlstyle{rm}
\isabellestyle{it}


\begin{document}

\title{Verification of the Deutsch-Schorr-Waite Graph Marking Algorithm using Data Refinement}

\author{Viorel Preoteasa and Ralph-Johan Back}

\maketitle

\begin{abstract}
The verification of the Deutsch-Schorr-Waite graph marking algorithm is used as a 
benchmark in many formalizations of pointer programs. The main purpose of this 
mechanization is to show how data refinement of invariant based programs
can be used in verifying practical algorithms. The verification starts
with an abstract algorithm working on a graph given by a relation {\em next}
on nodes. Gradually the abstract program is refined into Deutsch-Schorr-Waite
graph marking algorithm where only one bit per graph node of additional memory
is used for marking. 
\end{abstract}

\tableofcontents

\section{Introduction}
The verification of the Deutsch-Schorr-Waite (DSW) \cite{schorr:waite:1967,knuth:1997} 
graph marking algorithm is used as a benchmark in many formalizations
of pointer programs \cite{mehta:nipkow:2003,Abrial:2003}. The main purpose of this 
mechanization is to show how data refinement \cite{preoteasa:back:2009} of invariant 
based programs \cite{Back80:invariants,Back83:invariants,aBack08,back:preoteasa:2008}
can be used in verifying practical algorithms.

The DSW algorithm marks all nodes in a graph that are reachable from a {\em root} node. 
The marking is achieved using only one extra bit of memory for every node. The 
graph is given by two pointer functions, {\em left} and {\em right}, which for any given 
node return its left and right successors, respectively. While marking, the 
left and right functions are altered to represent a stack that describes the 
path from the root to the current node in the graph. On completion the original 
graph structure is restored. We construct the DSW algorithm by a sequence of 
three successive data refinement steps. One step in these refinements is
a generalization of the DSW algorithm to an algorithm which marks a
graph given by a family of pointer functions instead of left and right only.

Invariant based programming is an approach to construct correct programs 
where we start by identifying all basic situations (pre- and post-conditions, and loop 
invariants) that could arise during the execution of the algorithm. 
These situations are determined and described before any code is written. 
After that, we identify the transitions between the situations, which 
together determine the flow of control in the program. The transitions 
are verified at the same time as they are constructed. The correctness 
of the program is thus established as part of the construction process. 

Data refinement \cite{hoare:1972,back-1980,back:vonwright:2000,deroever:1999} 
is a technique of building correct programs working on concrete data structures 
as refinements of more abstract programs working on abstract data structures. 
The correctness of the final program follows from the correctness of the abstract 
program and from the correctness of the data refinement.

Both the semantics and the data refinement of invariant based programs were 
formalized in \cite{preoteasa:back:afp:2010}, and this verification is based on them.

We use a simple model of pointers where addresses (pointers, nodes) are 
the elements of a set and pointer fields are global pointer functions from
addresses to addresses. Pointer updates ($x.\mathit{left} := y$) are done
by modifying the global pointer function $\mathit{left} := \mathit{left}(x := y)$. 
Because of the nature of the marking algorithm where no allocation and 
disposal of memory are needed we do not treat these operations.

A number of Isabelle techniques are used here. The class mechanism is used
for extending the complete lattice theories as well as for introducing 
well founded and transitive relations. The polimorphism is used for the
state of the computation. In \cite{preoteasa:back:afp:2010} the state of computation 
was introduced as a type variable, or even more generaly, state predicates were introduced
as elements of a complete (boolean) lattice.
Here the state of the computation is instantiated with various tuples ranging
from the abstract data in the first algorithm to the concrete data
in the final refinement. The locale mechanism of Isabelle is used to
introduce the specification variables and their invariants. These
specification variables are used for example to prove that the main variables
are restored to their initial values when the algorithm terminates.
The locale extension and partial instantiation mechanisms turn out
to be also very useful in the data refinements of DSW. We start with
a locale which fixes the abstract graph as a relation {\em next} on
nodes. This locale is first partially interpreted into a locale
which replaces {\em next} by a union of a family of pointer functions. 
In the final refinement step the locale of the pointer functions is
interpreted into a locale with only two pointer functions, {\em left} and 
{\em right}.



\parindent 0pt\parskip 0.5ex

% generated text of all theories
\input{session}

% optional bibliography
\bibliographystyle{abbrv}
\bibliography{root}

\end{document}

%%% Local Variables:
%%% mode: latex
%%% TeX-master: t
%%% End:
