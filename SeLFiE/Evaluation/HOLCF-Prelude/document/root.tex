\documentclass[11pt,a4paper]{article}
\usepackage{isabelle,isabellesym}
\usepackage{xspace}

% further packages required for unusual symbols (see also
% isabellesym.sty), use only when needed

%\usepackage{amssymb}
  %for \<leadsto>, \<box>, \<diamond>, \<sqsupset>, \<mho>, \<Join>,
  %\<lhd>, \<lesssim>, \<greatersim>, \<lessapprox>, \<greaterapprox>,
  %\<triangleq>, \<yen>, \<lozenge>

%\usepackage[greek,english]{babel}
  %option greek for \<euro>
  %option english (default language) for \<guillemotleft>, \<guillemotright>

%\usepackage[only,bigsqcap]{stmaryrd}
  %for \<Sqinter>

%\usepackage{eufrak}
  %for \<AA> ... \<ZZ>, \<aa> ... \<zz> (also included in amssymb)

%\usepackage{textcomp}
  %for \<onequarter>, \<onehalf>, \<threequarters>, \<degree>, \<cent>,
  %\<currency>

% this should be the last package used
\usepackage{pdfsetup}

% urls in roman style, theory text in math-similar italics
\urlstyle{rm}
\isabellestyle{it}

% for uniform font size
%\renewcommand{\isastyle}{\isastyleminor}

\newcommand\hlint{\texttt{HLint}\xspace}

\begin{document}

\title{Isabelle/HOLCF-Prelude}
\author{%
  Joachim Breitner\thanks{Supported by the Deutsche Telekom Stiftung.},
  Brian Huffman,
  Neil Mitchell,
  and
  Christian Sternagel\thanks{Supported by the Austrian Science Fund (FWF): J3202.}}
\maketitle

\begin{abstract}
The Isabelle/HOLCF-Prelude is a formalization of a large part of Haskell's
standard prelude \cite{haskell-prelude} in Isabelle/HOLCF. We use it to
\begin{itemize}
\item prove the correctness of the Eratosthenes' Sieve, in its self-referential implementation commonly used to showcase Haskell's laziness,
\item prove correctness of GHC's ``fold/build'' rule and related rewrite rules, and
\item certify a number of hints suggested by \hlint.
\end{itemize}
The work was presented at HART 2013~\cite{hart2013}.
\end{abstract}

\tableofcontents


% sane default for proof documents
\parindent 0pt\parskip 0.5ex

% generated text of all theories
\input{session}

\section*{Acknowledgments}

We thank Lars Hupel for his help with the final AFP submission.

% optional bibliography
\bibliographystyle{abbrv}
\bibliography{root}

\end{document}

%%% Local Variables:
%%% mode: latex
%%% TeX-master: t
%%% End:

