\documentclass[11pt,a4paper]{article}
\usepackage{isabelle,isabellesym}

% this should be the last package used
\usepackage{pdfsetup}

% urls in roman style, theory text in math-similar italics
\urlstyle{rm}
\isabellestyle{it}

\newcommand\isafor{\textsf{IsaFoR}}
\newcommand\ceta{\textsf{Ce\kern-.18emT\kern-.18emA}}

\begin{document}

\title{Matrices, Jordan Normal Forms, and Spectral Radius Theory\footnote{Supported by FWF (Austrian Science Fund) project Y757.}}
\author{Ren\'e Thiemann and Akihisa Yamada}
\maketitle

\begin{abstract}
  Matrix interpretations are useful as measure functions in termination proving.
  In order to use these interpretations also for complexity analysis, 
  the growth rate of matrix powers has to examined. Here, we formalized
  an important result of spectral radius theory, namely that the growth rate
  is polynomially bounded if and only if the spectral radius of a matrix is at most one. 
  
  To formally prove this result we first studied the growth rates of matrices
  in Jordan normal form, and prove the result that every 
  complex matrix has a Jordan normal form by means of two algorithms:
  we first convert matrices into similar ones via Schur decomposition, and
  then apply a second algorithm which converts an upper-triangular matrix into
  Jordan normal form. We further showed uniqueness of Jordan normal forms which 
  then gives rise to a modular algorithm to compute individual blocks of a Jordan
  normal form. 
  
  The whole development is based on a new abstract type for matrices, which is
  also executable by a suitable setup of the code generator. It 
  completely subsumes our former AFP-entry on executable matrices 
  \cite{Matrix-AFP}, and its main advantage is its close connection to the
  HMA-representation which allowed us to easily adapt existing proofs on 
  determinants.
  
 All the results have been applied to improve \ceta\ \cite{CeTA,CeTAcomplexity},
 our certifier to validate termination and complexity proof certificates.
\end{abstract}

\tableofcontents

\section{Introduction}

The spectral radius of a square, complex valued matrix $A$ is defined as the 
largest norm of some eigenvalue $c$ with eigenvector $v$. 
It is a central notion to estimate how
the values in $A^n$ for increasing $n$. If the spectral radius is larger
than $1$, clearly the values grow exponentially, since then 
$A^n \cdot v = c^n \cdot v$ becomes exponentially large. 

The other results, namely
that the values in $A^n$ are bounded by a constant, 
if the spectral radius is smaller
than $1$, and that there is a polynomial bound if the spectral radius
is exactly $1$ are only immediate for matrices which have an eigenbasis,
a precondition which is not satisfied by every matrix.

However, these results are derivable via Jordan normal forms (JNFs): 
If $J$ is a JNF of $A$, then the growth rates of $A^n$ and $J^n$ are related
by a constant as $A$ and $J$ are similar matrices. And for the values in $J^n$ 
there is a closed formula which gives the desired complexity bounds.
To be more precise, the values in $J^n$ are bounded by 
${\cal O}(|c|^n \cdot n^{k-1})$ where $k$ is the size of the largest
block of an eigenvalue $c$ which has maximal norm w.r.t.\ the set of all
eigenvalues. And since every complex matrix has a JNF, we can derive the 
polynomial (resp.\ constant bounds), if the spectral radius is 1 (resp.\ smaller
than 1).

These results are already applied in current complexity tools, and the motivation
of this development was to extend our certifier \ceta\ to be able to validate
corresponding complexity proofs. To this end, we formalized
the  following main results:
\begin{itemize}
\item an algorithm to compute the characteristic polynomial, since
  the eigenvalues are exactly the roots of this polynomial;
\item the complexity bounds for JNFs; and
\item an algorithm which computes JNFs for every matrix, provided that the
  list of eigenvalues is given. With the help of the fundamental theorem
  of algebra this shows that every complex matrix has a JNF.
\end{itemize}

Since \ceta\ is generated from Isabelle/HOL via code-generation, all the 
algorithms and results need to be available at code-generation time. Especially
there is no possibility to create types on the fly which are chosen to fit 
the matrix dimensions of the input. To this end, we cannot use the 
matrix-representation of HOL multivariate analysis (HMA).

Instead, we provide a new matrix library which is based on HOL-algebra with
its explicit carriers. In contrast to our earlier development \cite{Matrix-AFP}, 
we do not immediately formalize everything as lists of lists, 
but use a more mathematical
notion as triples of the form (dimension, dimension, characteristic-function).
This makes reasoning very similar to HMA, and a suitable implementation type
can be chosen afterwards: we provide one via immutable arrays (we use IArray's from
the HOL library),
but one can also think of an implementation for sparse matrices, etc.
Even the infinite carrier itself is executable where we rely upon Lochbihler's
container framework \cite{Containers-AFP} to have different set representations 
at the same time.

As a consequence of not using HMA, we could not directly reuse existing 
algorithms which
have been formalized for this representation. For instance, we formalized our
own version of Gauss-Jordan elimination which is not very different to the
one of Divas\'on and Aransay in \cite{Gauss_Jordan-AFP}: both define row-echelon
form and apply elementary row transformations. Whereas Gauss-Jordan elimination
has been developed from scratch as a case-study to see how suitable our matrix
representation is, in other cases we often just
copied and adjusted existing proofs from HMA. For instance, most of the library for 
determinants has been copied from the Isabelle distribution and adapted to our
matrix representation.


As a result of our formalization, \ceta\ is now able to check  
polynomial bounds for matrix interpretations \cite{MatrixJAR}. 


% include generated text of all theories
\input{session}



\bibliographystyle{abbrv}
\bibliography{root}

\end{document}
