\documentclass[11pt,a4paper]{article}
\usepackage{isabelle,isabellesym}

\usepackage{amssymb}
% this should be the last package used
\usepackage{pdfsetup}

% urls in roman style, theory text in math-similar italics
\urlstyle{rm}
\isabellestyle{it}

\newcommand\isafor{\textsf{IsaFoR}}
\newcommand\ceta{\textsf{Ce\kern-.18emT\kern-.18emA}}
\begin{document}

\title{Abstract Rewriting}
\author{Christian Sternagel and Ren\'e Thiemann}
\maketitle

\begin{abstract}
We present an Isabelle formalization of abstract rewriting (see, e.g.,
\cite{BaaderNipkow}). First, we define standard relations like
\emph{joinability}, \emph{meetability}, \emph{conversion}, etc. Then, we
formalize important properties of abstract rewrite systems, e.g.,
confluence and strong normalization. Our main concern is on strong
normalization, since this formalization is the basis of \cite{CeTA} (which
is mainly about strong normalization of term rewrite systems; see also
\isafor/\ceta's
website\footnote{\url{http://cl-informatik.uibk.ac.at/software/ceta}}).
Hence lemmas involving strong normalization, constitute by far the biggest
part of this theory. One of those is Newman's lemma.
\end{abstract}

\tableofcontents

A description of this formalization will be available in
\cite{Sternagel2010}.

% include generated text of all theories
\input{session}

\bibliographystyle{abbrv}
\bibliography{root}
\end{document}
