\documentclass[11pt,a4paper]{report}
\usepackage{isabelle,isabellesym}
\usepackage{railsetup}
\usepackage{amssymb}
\usepackage[english]{babel}
\usepackage{wasysym}

% this should be the last package used
\usepackage{pdfsetup}

% urls in roman style, theory text in math-similar italics
\urlstyle{rm}
\isabellestyle{it}

\renewcommand{\isamarkupcmt}[1]{\hangindent5ex{\isastylecmt --- #1}}


\begin{document}

\title{The Imperative Refinement Framework} 

\author{Peter Lammich}
\maketitle

\begin{abstract}
  We present the Imperative Refinement Framework (IRF), a tool that supports a
  stepwise refinement based approach to imperative programs. This entry is
  based on the material we presented in [ITP-2015, CPP-2016].

  It uses the Monadic Refinement Framework as a frontend for the specification of 
  the abstract programs, and Imperative/HOL as a backend to generate executable 
  imperative programs. 

  The IRF comes with tool support to synthesize 
  imperative programs from more abstract, functional ones, using efficient 
  imperative implementations for the abstract data structures.

  This entry also includes the Imperative Isabelle Collection Framework (IICF), which 
  provides a library of re-usable imperative collection data structures.

  Moreover, this entry contains a quickstart guide and a reference manual, which provide
  an introduction to using the IRF for Isabelle/HOL experts. 
  It also provids a collection of (partly commented) practical examples,
  some highlights being Dijkstra's Algorithm, Nested-DFS, and a generic worklist algorithm 
  with subsumption.

  Finally, this entry contains benchmark scripts that compare the runtime of some examples
  against reference implementations of the algorithms in Java and C++. 
  
  
  \vfill
  {\footnotesize
  \begin{description}
  \item[{[{ITP-2015}]}] Peter Lammich: Refinement to Imperative/HOL. ITP 2015: 253--269
  \item[{[{CPP-2016}]}] Peter Lammich: Refinement based verification of imperative data structures. CPP 2016: 27--36
  \end{description}
  }

\end{abstract}

\clearpage

\tableofcontents

\clearpage

% sane default for proof documents
\parindent 0pt\parskip 0.5ex

% generated text of all theories
\input{session}

% optional bibliography
%\bibliographystyle{abbrv}
%\bibliography{root}

\end{document}

%%% Local Variables:
%%% mode: latex
%%% TeX-master: t
%%% End:
