\documentclass[11pt,a4paper]{article}
\usepackage{isabelle,isabellesym}
\usepackage{amsfonts, amsmath, amssymb}

% this should be the last package used
\usepackage{pdfsetup}

% urls in roman style, theory text in math-similar italics
\urlstyle{rm}
\isabellestyle{it}


\begin{document}

\title{Probabilistic Primality Testing}
\author{Daniel Stüwe and Manuel Eberl}
\maketitle

\begin{abstract}
The most efficient known primality tests are \emph{probabilistic} in the sense that they use randomness and may, with some probability, mistakenly classify a composite number as prime -- but never a prime number as composite. Examples of this are the Miller--Rabin test, the Solovay--Strassen test, and (in most cases) Fermat's test. 

This entry defines these three tests and proves their correctness. It also develops some of the number-theoretic foundations, such as Carmichael numbers and the Jacobi symbol with an efficient executable algorithm to compute it.
\end{abstract}

\newpage
\tableofcontents
\newpage
\parindent 0pt\parskip 0.5ex

\input{session}

\bibliographystyle{abbrv}
\bibliography{root}

\end{document}

%%% Local Variables:
%%% mode: latex
%%% TeX-master: t
%%% End:
