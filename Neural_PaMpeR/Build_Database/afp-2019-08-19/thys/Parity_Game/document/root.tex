\documentclass[11pt,a4paper]{scrartcl}
\usepackage{isabelle,isabellesym}
\usepackage{amsfonts,amssymb}
\usepackage{xspace}
\usepackage[utf8]{inputenc}
\usepackage[T1]{fontenc}

\typearea{11}
\renewcommand{\bf}{\normalfont\bfseries}
\renewcommand{\rm}{\normalfont\rmfamily}
\renewcommand{\it}{\normalfont\itshape}

\usepackage{pdfsetup}

% urls in roman style, theory text in math-similar italics
\urlstyle{rm}
\isabellestyle{it}

% for uniform font size
%\renewcommand{\isastyle}{\isastyleminor}

\newcommand{\Even}{\textsc{Even}\xspace}
\newcommand{\Odd}{\textsc{Odd}\xspace}

\begin{document}

\title{Positional Determinacy of Parity Games}
\author{Christoph Dittmann\\christoph.dittmann@tu-berlin.de}
\date{\today}
\maketitle

\begin{abstract}
  We present a formalization of parity games (a two-player game on
  directed graphs) and a proof of their positional determinacy in
  Isabelle/HOL.  This proof works for both finite and infinite games.
  We follow the proof in \cite{kreutzer2015}, which is based on
  \cite{zielonka1998}.
\end{abstract}

\tableofcontents
\newpage

\section{Introduction}

Parity games are games played by two players, called \Even and \Odd,
on labelled directed graphs.  Each node is labelled with their player
and with a natural number, called its \emph{priority}.

To call this a \emph{parity game}, we only need to assume that the
number of different priorities is finite.  Of course, this condition
is only relevant on infinite graphs.

One reason parity games are important is that determining the winner
is polynomial-time equivalent to the model-checking problem of the
modal $\mu$-calculus, a logic able to express LTL and CTL* properties
(\cite{bradfield2007}).

\subsection{Formal Introduction}

Formally, a parity game is $G = (V,E,V_0,\omega)$, where $(V,E)$ is a
directed graph, $V_0 \subseteq V$ is the set of \Even nodes, and
$\omega: V \to \mathbb{N}$ is a function with $|f(V)| < \infty$.

A \emph{play} is a maximal path in $G$.  A finite play is winning for
\Even iff the last node is not in $V_0$.  An infinite play is winning
for \Even iff the minimum priority occurring infinitely often on the
path is even.  On an infinite path at least one priority occurs
infinitely often because there is only a finite number of different
priorities.

A node $v$ is \emph{winning} for a player~$p$ iff all plays starting
from $v$ are winning for~$p$.  It is well-known that parity games are
\emph{determined}, that is, every node is winning for some player.

A more surprising property is that parity games are also
\emph{positionally determined}.  This means that for every node $v$
winning for \Even, there is a function $\sigma: V_0 \to V$ such that
all \Even needs to do in order to win from $v$ is to consult this
function whenever it is his turn (similarly if $v$ is winning for
\Odd).  This is also called a \emph{positional strategy} for the
winning player.

We define the \emph{winning region} of player~$p$ as the set of nodes
from which player~$p$ has positional winning strategies.  Positional
determinacy then says that the winning regions of \Even and of \Odd
partition the graph.

See \cite{automata2002/kuesters} for a modern survey on positional
determinacy of parity games.  Their proof is based on a proof by
Zielonka \cite{zielonka1998}.

\subsection{Overview}

Here we formalize the proof from \cite{kreutzer2015} in Isabelle/HOL.
This proof is similar to the proof in \cite{automata2002/kuesters},
but we do not explicitly define so-called ``$\sigma$-traps''.  Using
$\sigma$-traps could be worth exploring, because it has the potential
to simplify our formalization.

Our proof has no assumptions except those required by every parity
game.  In particular the parity game
\begin{itemize}
\item may have arbitrary cardinality,
\item may have loops,
\item may have deadends, that is, nodes with no successors.
\end{itemize}

The main theorem is in section \ref{subsec:positional_determinacy}.

\subsection{Technical Aspects}

We use a coinductive list of nodes to represent paths in a graph
because this gives us a uniform representation for finite and infinite
paths.  We can then express properties such as that a path is maximal
or conforms to a given strategy directly as coinductive properties.
We use the coinductive list developed by Lochbihler in
\cite{Coinductive-AFP}.

We also explored representing paths as functions \isa{nat\
  {\isasymRightarrow}\ {\isacharprime}a\ option} with the property
that the domain is an initial segment of \isa{nat} (and where
\isa{{\isacharprime}a} is the node type).  However, it turned out that
coinductive lists give simpler proofs.

It is possible to represent a graph as a function
\isa{{\isacharprime}a\ {\isasymRightarrow}\ {\isacharprime}a\
  {\isasymRightarrow}\ bool}, see for example in the proof of König's
lemma in \cite{Coinductive-AFP}.  However, we instead go for a record
which contains a set of nodes and a set of edges explicitly.  By not
requiring that the set of nodes is \isa{UNIV\ ::\ {\isacharprime}a\
  set} but rather a subset of \isa{UNIV\ ::\ {\isacharprime}a\ set},
it becomes easier to reason about subgraphs.

Another point is that we make extensive use of locales, in particular
to represent maximal paths conforming to a specific strategy.  Thus
proofs often start with \isa{\isacommand{interpret}\
  vmc{\isacharunderscore}path\ G\ P\ \ensuremath{v_0}\ p\
  \isasymsigma} to say that $P$ is a valid maximal path in the graph
$G$ starting in $v_0$ and conforming to the strategy $\sigma$ for
player $p$.

% sane default for proof documents
\parindent 0pt\parskip 0.5ex

% generated text of all theories
\input{session}

\clearpage
\phantomsection
\addcontentsline{toc}{section}{Bibliography}
\bibliographystyle{plain}
\bibliography{root}

\end{document}

%%% Local Variables:
%%% mode: latex
%%% TeX-master: t
%%% End:
