\documentclass[11pt,a4paper]{article}
\usepackage{isabelle,isabellesym}
\usepackage{amssymb,amsmath}
\usepackage[english]{babel}
\usepackage[only,bigsqcap]{stmaryrd}
\usepackage{wasysym}
\usepackage{booktabs}

% this should be the last package used
\usepackage{pdfsetup}

% urls in roman style, theory text in math-similar italics
\urlstyle{rm}
\isabellestyle{it}

\begin{document}

\title{Monad Normalisation}
\author{Joshua Schneider and Manuel Eberl and Andreas Lochbihler}
\maketitle

\begin{abstract}
  The usual monad laws can directly be used as rewrite rules for Isabelle's simplifier to normalise
  monadic HOL terms and decide equivalences.  In a commutative monad, however, the commutativity
  law is a higher-order permutative rewrite rule that makes the simplifier loop.  This AFP entry
  implements a simproc that normalises monadic expressions in commutative monads using ordered
  rewriting.  The simproc can also permute computations across control operators like \textit{if}
  and \textit{case}.
\end{abstract}


\tableofcontents

% sane default for proof documents
\parindent 0pt\parskip 0.5ex

% generated text of all theories
\input{session}

% optional bibliography
\bibliographystyle{abbrv}
\bibliography{root}

\end{document}
