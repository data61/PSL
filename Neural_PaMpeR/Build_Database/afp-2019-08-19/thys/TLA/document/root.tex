\documentclass[11pt,a4paper]{article}
%\usepackage{amssymb, amsfonts}
%\usepackage{hyperref}
\usepackage{wasysym}
\usepackage{isabelle,isabellesym}

\usepackage{amssymb}
\usepackage[only,bigsqcap]{stmaryrd}
\usepackage{textcomp}

% this should be the last package used
\usepackage{pdfsetup}

% urls in roman style, theory text in math-similar italics
\urlstyle{rm}
\isabellestyle{it}

\newcommand{\tlastar}{TLA$^{*}$}

\begin{document}

\title{A Definitional Encoding of TLA in Isabelle/HOL}
\author{Gudmund Grov \& Stephan Merz}
\maketitle

\begin{abstract}
We mechanise the logic \tlastar{} \cite{Merz99}, an extension of Lamport's Temporal Logic of Actions (TLA) \cite{Lamport94} for specifying 
and reasoning about concurrent and reactive systems. Aiming at a framework for mechanising the verification
of TLA (or \tlastar{}) specifications, this contribution reuses some elements from a
previous axiomatic encoding of TLA in Isabelle/HOL by the second author \cite{Merz98}, which has been part of the
Isabelle distribution. In contrast to that previous work, we give here a shallow, definitional
embedding, with the following highlights:
\begin{itemize}
\item a theory of infinite sequences, including a formalisation of the concepts of stuttering invariance central to TLA and TLA*;
\item a definition of the semantics of TLA*, which extends TLA by a mutually-recursive definition of formulas and pre-formulas, generalising TLA action formulas;
\item a substantial set of derived proof rules, including the TLA* axioms and Lamport's proof rules for system verification;
\item a set of examples illustrating the usage of Isabelle/TLA* for reasoning about systems.
\end{itemize}
Note that this work is unrelated to the ongoing development of a proof system for the specification language TLA+, 
which includes an encoding of TLA+ as a new Isabelle object logic \cite{chaudhuri:tlaps}.

A previous version of this embedding has been used heavily in the work described in \cite{Grov09}. 
\end{abstract}

\tableofcontents

% sane default for proof documents
\parindent 0pt\parskip 0.5ex

% generated text of all theories
\input{session}

% optional bibliography
\bibliographystyle{abbrv}
\bibliography{root}

\end{document}

%%% Local Variables:
%%% mode: latex
%%% TeX-master: t
%%% End:
