\documentclass[11pt,a4paper]{article}
\usepackage{isabelle,isabellesym}
\usepackage{amsfonts, amsmath, amssymb}

% this should be the last package used
\usepackage{pdfsetup}

% urls in roman style, theory text in math-similar italics
\urlstyle{rm}
\isabellestyle{it}


\begin{document}

\title{The Euler--MacLaurin summation formula}
\author{Manuel Eberl}
\maketitle

\begin{abstract}
The Euler--MacLaurin formula relates the value of a discrete sum 
$\sum_{i=a}^b f(i)$ to that of the integral $\int_a^b f(x)\,\text{d}x$ in 
terms of the derivatives of $f$ at $a$ and $b$ and a remainder term. Since the remainder term is often very small as $b$ grows, this can be used 
to compute asymptotic expansions for sums.
	
This entry contains a proof of this formula for functions from the reals to an arbitrary Banach space. Two variants of the formula are given: the standard textbook version and a variant outlined in \emph{Concrete Mathematics}~\cite{GKP_CM} that is more useful for deriving asymptotic estimates. 

As example applications, we use that formula to derive the full asymptotic expansion of the harmonic numbers and the sum of inverse squares.
\end{abstract}

\tableofcontents
\newpage
\parindent 0pt\parskip 0.5ex

\input{session}

\bibliographystyle{abbrv}
\bibliography{root}

\end{document}

%%% Local Variables:
%%% mode: latex
%%% TeX-master: t
%%% End:
