 \documentclass[11pt,a4paper]{article}
\usepackage{isabelle,isabellesym}

% further packages required for unusual symbols (see also
% isabellesym.sty), use only when needed

%\usepackage{amssymb}
  %for \<leadsto>, \<box>, \<diamond>, \<sqsupset>, \<mho>, \<Join>,
  %\<lhd>, \<lesssim>, \<greatersim>, \<lessapprox>, \<greaterapprox>,
  %\<triangleq>, \<yen>, \<lozenge>

%\usepackage{eurosym}
  %for \<euro>

\usepackage[only,bigsqcap]{stmaryrd}
  %for \<Sqinter>

%\usepackage{eufrak}
  %for \<AA> ... \<ZZ>, \<aa> ... \<zz> (also included in amssymb)

%\usepackage{textcomp}
  %for \<onequarter>, \<onehalf>, \<threequarters>, \<degree>, \<cent>,
  %\<currency>

% this should be the last package used
\usepackage{pdfsetup}

% urls in roman style, theory text in math-similar italics
\urlstyle{rm}
\isabellestyle{it}

% for uniform font size
%\renewcommand{\isastyle}{\isastyleminor}


\begin{document}

\title{Transformer Semantics}
\author{Georg Struth}
\maketitle

\begin{abstract}
  These mathematical components formalise predicate transformer
  semantics for programs, yet currently only for partial correctness
  and in the absence of faults.  A first part for isotone (or
  monotone), Sup-preserving and Inf-preserving transformers follows
  Back and von Wright's approach, with additional emphasis on the
  quantalic structure of algebras of transformers.  The second part
  develops Sup-preserving and Inf-preserving predicate transformers
  from the powerset monad, via its Kleisli category and
  Eilenberg-Moore algebras, with emphasis on adjunctions and
  dualities, as well as isomorphisms between relations, state
  transformers and predicate transformers.
\end{abstract}

\tableofcontents

\section{Introductory Remarks}

Predicate transformers yield standard denotational semantics for
imperative programs; they have been investigated for around fifty
years and are widely used in program verification.  These components
provide yet another take on this topic with Isabelle (previous
formalisations in the AFP
include~\cite{Preoteasa11b,GomesGHSW16,GomesS16}).

The first part, like Preoteasa's work~\cite{Preoteasa11b}, follows by
and large Back and von Wright's seminal monograph~\cite{BackvW98}.
Isotone (or monotone), sup-preserving and inf-preserving transformers
are developed in a categorical setting as morphisms of orderings and
complete lattices.  The approach is type-driven; concepts are usually
formalised with the most general suitable types. Due to this, the
algebras of transformers cannot be captured within Isabelle's type
classes or locales. They describe algebraic properties of typed
function spaces (enriched homsets of categories of complete lattices)
in terms of typed quantales or quantaloids~\cite{Rosenthal90}. Special
focus is on notions of recursion and iteration in this typed
setting. In particular, propositional Hoare logics and basic
refinement calculi---for partial correctness and without assignment
laws---are derived. For transformers that are endofunctions, instance
proofs for quantales are given. This brings theorems about quantales
and from the Kleene algebra hierarchy into scope.

Based on this, the second part presents an alternative, more detailed
development with sets. It starts from the monad of the powerset
functor, its Kleisli category and its Eilenberg-Moore algebras; a view
that has been promoted, for instance, by
Jacobs~\cite{Jacobs17}. General monads cannot be handled by Isabelle's
type system, only particular instances can be formalised---at the
level of exercises in category theory textbooks. With this approach,
binary relations, state transformers modelled as arrows of the Kleisli
category of the powerset monad, and predicate transformer algebras,
Sup-lattices which arise as Eilenberg-Moore algebras of the powerset
monad, are related like in Jacob's state-effect triangles. In
particular, the isomorphisms between the quantalic structure of
relations, that of state transformers and that of various predicate
transformers is spelled out in detail. In addition, the symmetries and
dualities between four kinds of predicate transformers (forward and
backward modal box and diamond operators in the parlance of dynamic
logic) are formalised. Beyond that, the quantalic structure of state
transformers is detailed first in a typed setting, and secondly in a
single-typed one, where state transformers are shown to form quantales
and hence Kleene algebras.

It should be straightforward to integrate these mathematical
components into verification components along the lines
of~\cite{ArmstrongGS16,GomesS16}.  Beyond that, an integration with
the predicate transformers obtained from modal Kleene
algebras~\cite{GomesGHSW16} seems interesting for verification
applications.  Possible extensions and refinements include the
development of verification conditions for recursion beyond those for
while-loops, approaches to total correctness and fault semantics, more
complete (re)encodings of Back and von Wright's approach,
formalisations of domain theory, links between isotone transformers
and Isabelle components for multirelational
semantics~\cite{FurusawaS15} and extensions to probabilistic
transformers~\cite{McIverM05}.

% sane default for proof documents
\parindent 0pt\parskip 0.5ex

% generated text of all theories
\input{session}

% optional bibliography
\bibliographystyle{abbrv}
\bibliography{root}

\end{document}

%%% Local Variables:
%%% mode: latex
%%% TeX-master: t
%%% End:
