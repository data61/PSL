\documentclass[11pt,a4paper]{article}
\usepackage{isabelle,isabellesym}

\usepackage{amssymb}
\usepackage[english]{babel}

% this should be the last package used
\usepackage{pdfsetup}

% urls in roman style, theory text in math-similar italics
\urlstyle{rm}
\isabellestyle{it}


\begin{document}

\title{Converting Linear-Time Temporal Logic to Generalized B\"uchi Automata}
\author{Alexander Schimpf and Peter Lammich}
\maketitle

\begin{abstract}
We formalize linear-time temporal logic (LTL) and the algorithm by Gerth et al.\ to convert LTL formulas to generalized B\"uchi automata.
We also formalize some syntactic rewrite rules that can be applied to optimize the LTL formula before conversion.
Moreover, we integrate the Stuttering Equivalence AFP-Entry by Stefan Merz, adapting the lemma that next-free LTL formula cannot distinguish between
stuttering equivalent runs to our setting.

We use the Isabelle Refinement and Collection framework, as well as the Autoref tool, to obtain a refined version of our algorithm,
from which efficiently executable code can be extracted.
\end{abstract}

\clearpage

\tableofcontents
\clearpage

% sane default for proof documents
\parindent 0pt\parskip 0.5ex

\chapter{Introduction}
  This development provides an efficient, extensible, machine checked collections framework for use
  in Isabelle/HOL. The library adopts the concepts of interface, implementation and generic algorithm
  known from object oriented (collection) libraries like the C++ Standard Template Library\cite{C++STL} or 
  the Java Collections Framework\cite{JavaCollFr} and makes them available in the Isabelle/HOL environment.

  The library uses data refinement techniques to refine an abstract specification (in terms of high-level concepts such as sets) to a more concrete implementation (based on collection datastructures like red-black-trees).
  This allows algorithms to be proven on the abstract level at which proofs are simpler because they are not cluttered with low-level details.

  The code-generator of Isabelle/HOL can be used to generate efficient code in all supported target languages, i.e. Haskell, SML, and OCaml.

  For more documentation and introductory material refer to the userguide (Section~\ref{thy:Userguide}) and the ITP-2010 paper \cite{LammichLochbihler2010ITP}.


% generated text of all theories
\input{session}

% optional bibliography
\bibliographystyle{abbrv}
\bibliography{root}

\end{document}

%%% Local Variables:
%%% mode: latex
%%% TeX-master: t
%%% End:
