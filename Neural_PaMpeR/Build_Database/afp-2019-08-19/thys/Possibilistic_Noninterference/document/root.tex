\documentclass[11pt,a4paper]{article}
\usepackage{isabelle,isabellesym}

%\usepackage{graphicx}
%\usepackage{alltt}  
%\usepackage{url}
%\usepackage{hyperref}
%\usepackage{mathptmx}
%\include{myCommands}
%\include{tikzlib}
\usepackage[all]{xy}
%\usepackage{cite}

\usepackage[only,bigsqcap]{stmaryrd}

% this should be the last package used
\usepackage{pdfsetup}

% urls in roman style, theory text in math-similar italics
\urlstyle{rm}
\isabellestyle{it}


\begin{document}

\title{Possibilistic Noninterference\thanks{Supported by the DFG project Ni 491/13--1 (part of the DFG priority program RS3) and the DFG RTG 1480.}
}
\author{Andrei Popescu \hspace*{10ex} Johannes H\"{o}lzl}
%\institute{1: Technische Universit\"{a}t M\"{u}nchen 
%           \hspace*{2ex} 2: Institute of Mathematics Simion Stoilow, Romania          
%           }

\maketitle

\begin{abstract}
We formalize a wide variety of Volpano/Smith-style 
noninterference notions for a while language with parallel composition.  
We systematize and classify these notions according to compositionality w.r.t.~the language constructs.  
Compositionality yields sound syntactic criteria (a.k.a.~type systems) in a uniform way.  
\end{abstract}


\tableofcontents

% sane default for proof documents
\parindent 0pt\parskip 0.5ex

\chapter{Introduction}
  This development provides an efficient, extensible, machine checked collections framework for use
  in Isabelle/HOL. The library adopts the concepts of interface, implementation and generic algorithm
  known from object oriented (collection) libraries like the C++ Standard Template Library\cite{C++STL} or 
  the Java Collections Framework\cite{JavaCollFr} and makes them available in the Isabelle/HOL environment.

  The library uses data refinement techniques to refine an abstract specification (in terms of high-level concepts such as sets) to a more concrete implementation (based on collection datastructures like red-black-trees).
  This allows algorithms to be proven on the abstract level at which proofs are simpler because they are not cluttered with low-level details.

  The code-generator of Isabelle/HOL can be used to generate efficient code in all supported target languages, i.e. Haskell, SML, and OCaml.

  For more documentation and introductory material refer to the userguide (Section~\ref{thy:Userguide}) and the ITP-2010 paper \cite{LammichLochbihler2010ITP}.


% generated text of all theories
\input{session}

% optional bibliography
\bibliographystyle{abbrv}
\bibliography{root}

\end{document}

%%% Local Variables:
%%% mode: latex
%%% TeX-master: t
%%% End:
