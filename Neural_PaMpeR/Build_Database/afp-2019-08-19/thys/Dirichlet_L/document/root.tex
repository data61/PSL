\documentclass[11pt,a4paper]{article}
\usepackage{isabelle,isabellesym}
\usepackage{amsfonts, amsmath, amssymb}

% this should be the last package used
\usepackage{pdfsetup}

% urls in roman style, theory text in math-similar italics
\urlstyle{rm}
\isabellestyle{it}


\begin{document}

\title{Dirichlet $L$-functions and Dirichlet's Theorem}
\author{Manuel Eberl}
\maketitle

\begin{abstract}
This article provides a formalisation of Dirichlet characters and Dirichlet $L$-functions including proofs of their basic properties -- most notably their analyticity, their areas of convergence, and their non-vanishing for $\mathfrak{R}(s)\geq 1$. All of this is built in a very high-level style using Dirichlet series. The proof of the non-vanishing follows a very short and elegant proof by Newman~\cite{newman1998analytic}, which we attempt to reproduce faithfully in a similar level of abstraction in Isabelle.

This also leads to a relatively short proof of Dirichlet's Theorem, which states that, if $h$ and $n$ are coprime, there are infinitely many primes $p$ with $p \equiv h \pmod{n}$.
\end{abstract}

\newpage
\tableofcontents
\newpage
\parindent 0pt\parskip 0.5ex

\input{session}

\bibliographystyle{abbrv}
\bibliography{root}

\end{document}

%%% Local Variables:
%%% mode: latex
%%% TeX-master: t
%%% End:
