\documentclass[11pt,a4paper]{article}
\usepackage{isabelle,isabellesym}
\usepackage{amsfonts, amsmath, amssymb}

% this should be the last package used
\usepackage{pdfsetup}

% urls in roman style, theory text in math-similar italics
\urlstyle{rm}
\isabellestyle{it}


\begin{document}

\title{Pell's Equation}
\author{Manuel Eberl}
\maketitle

\begin{abstract}
This article gives the basic theory of Pell's equation $x^2 = 1 + D y^2$, where $D\in\mathbb{N}$ is a parameter and $x$, $y$ are integer variables.

The main result that is proven is the following: If $D$ is not a perfect square, then there exists a \emph{fundamental solution} $(x_0, y_0)$ that is not the trivial solution $(1, 0)$ and which generates all other solutions $(x, y)$ in the sense that there exists some $n\in\mathbb{N}$ such that $|x| + |y| \sqrt{D} = (x_0 + y_0 \sqrt{D})^n$. This also implies that the set of solutions is infinite, and it gives us an explicit and executable characterisation of all the solutions.

Based on this, simple executable algorithms for computing the fundamental solution and the infinite sequence of all non-negative solutions are also provided.
\end{abstract}

\newpage
\tableofcontents
\newpage
\parindent 0pt\parskip 0.5ex

\input{session}

\bibliographystyle{abbrv}
\bibliography{root}

\end{document}

%%% Local Variables:
%%% mode: latex
%%% TeX-master: t
%%% End:
