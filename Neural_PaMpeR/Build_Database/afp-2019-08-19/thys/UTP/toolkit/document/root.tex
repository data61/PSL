\documentclass[11pt,a4paper]{article}
\usepackage{isabelle,isabellesym}
\usepackage{fullpage}
\usepackage[usenames,dvipsnames]{color}
\usepackage{document}

% further packages required for unusual symbols (see also
% isabellesym.sty), use only when needed

\usepackage{amssymb}
  %for \<leadsto>, \<box>, \<diamond>, \<sqsupset>, \<mho>, \<Join>,
  %\<lhd>, \<lesssim>, \<greatersim>, \<lessapprox>, \<greaterapprox>,
  %\<triangleq>, \<yen>, \<lozenge>

\usepackage[english]{babel}
  %option greek for \<euro>
  %option english (default language) for \<guillemotleft>, \<guillemotright>

\usepackage{stmaryrd}
  %for \<Sqinter>

\usepackage{eufrak}
  %for \<AA> ... \<ZZ>, \<aa> ... \<zz> (also included in amssymb)

%\usepackage{textcomp}
  %for \<onequarter>, \<onehalf>, \<threequarters>, \<degree>, \<cent>,
  %\<currency>

% this should be the last package used
\usepackage{pdfsetup}

% urls in roman style, theory text in math-similar italics
\urlstyle{rm}
\isabellestyle{it}

% for uniform font size
%\renewcommand{\isastyle}{\isastyleminor}

\begin{document}

\title{Mathematical Toolkit for Isabelle/UTP}

\author{Simon Foster \and Pedro Ribeiro \and Frank Zeyda}

\maketitle

\begin{abstract}
This document describes our mathematical toolkit for Isabelle/UTP, which provides a foundational
collection of definition, theorems, and proof facilities. This includes extensions to existing
HOL libraries, such as for list and partial functions, and also new type definitions, theorems, and 
Isabelle/HOL commands.
\end{abstract}

\tableofcontents

% sane default for proof documents
\parindent 0pt\parskip 0.5ex

\section{Introduction}

This document contains the description of our mathematical toolkit for 
Isabelle/UTP~\cite{Feliachi2010,Foster14c,Foster16a,Foster16c}, a mechanisation of Hoare and He's 
\emph{Unifying Theories of Programming}~\cite{Hoare&98,Cavalcanti&06}. The toolkit provides a 
foundational collection of additional HOL theorems, new abstract types, and proof facilities, upon 
which Isabelle/UTP depends. In brief, the toolkit contains the following principal items:

\begin{itemize}
  \item additional laws and functions for the list, map (partial functions), countable set, and finite set types;
  \item type definitions for partial and finite functions, together with additional functions and laws
    derived from the Z mathematical toolkit~\cite{zrm};
  \item positive subtypes of existing types;
  \item infinite sequences;
  \item the ``total recall'' package, which allows us to precisely control overriding of existing syntax 
    annotations.
\end{itemize}

A few other theories exist that add smaller utilities and additional laws.

% generated text of all theories
\input{session}

% optional bibliography
\bibliographystyle{abbrv}
\bibliography{root}

\end{document}
