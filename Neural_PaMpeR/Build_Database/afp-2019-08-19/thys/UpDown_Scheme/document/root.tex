\documentclass[11pt,a4paper]{article}

\usepackage{german}
\usepackage{isabelle,isabellesym}

% urls in roman style, theory text in math-similar italics
% \urlstyle{rm}
\isabellestyle{it}

\title{Verification of the \texttt{UpDown} scheme}
\author{Johannes H{\"o}lzl}

\begin{document}
\maketitle

\begin{abstract}

The \texttt{UpDown} scheme is a recursive scheme used to compute the stiffness matrix on a special form of
sparse grids. Usually, when discretizing a Euclidean space of dimension $d$ we need $O(n^d)$ points,
for $n$ points along each dimension. Sparse grids are a hierarchical representation where the number
of points is reduced to $O(n\cdot\log(n)^d)$. One disadvantage of such sparse grids is that the
algorithm now operate recursively in the dimensions and levels of the sparse grid.

The \texttt{UpDown} scheme allows us to compute the stiffness matrix on such a sparse grid. The stiffness
matrix represents the influence of each representation function on the $L^2$ scalar product.
For a detailed description see Pfl{\"u}ger's PhD thesis~\cite{pflueger10spatially}. This formalization
was developed as an interdisciplinary project (IDP) at the TU~M{\"u}nchen~\cite{hoelzl09updown}.

\end{abstract}

\textbf{Note:} This development has two main theories. The correctnes of the UpDown scheme, and
a verification of an imperative version of it. Both theories can not be merged, as they use
different orders on the product type.

\tableofcontents

% sane default for proof documents
\parindent 0pt\parskip 0.5ex

% generated text of all theories
\input{session}

% optional bibliography
\bibliographystyle{abbrv}
\bibliography{root}

\end{document}

%%% Local Variables:
%%% mode: latex
%%% TeX-master: t
%%% End:
