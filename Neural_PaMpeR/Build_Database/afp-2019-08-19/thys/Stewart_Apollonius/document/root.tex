\documentclass[11pt,a4paper]{article}
\usepackage{isabelle,isabellesym}

% further packages required for unusual symbols (see also
% isabellesym.sty), use only when needed

%\usepackage{amssymb}
  %for \<leadsto>, \<box>, \<diamond>, \<sqsupset>, \<mho>, \<Join>,
  %\<lhd>, \<lesssim>, \<greatersim>, \<lessapprox>, \<greaterapprox>,
  %\<triangleq>, \<yen>, \<lozenge>

%\usepackage{eurosym}
  %for \<euro>

%\usepackage[only,bigsqcap]{stmaryrd}
  %for \<Sqinter>

%\usepackage{eufrak}
  %for \<AA> ... \<ZZ>, \<aa> ... \<zz> (also included in amssymb)

%\usepackage{textcomp}
  %for \<onequarter>, \<onehalf>, \<threequarters>, \<degree>, \<cent>,
  %\<currency>

% this should be the last package used
\usepackage{pdfsetup}

% urls in roman style, theory text in math-similar italics
\urlstyle{rm}
\isabellestyle{it}

% for uniform font size
%\renewcommand{\isastyle}{\isastyleminor}

\begin{document}

\title{Stewart's Theorem and Apollonius' Theorem}
\author{Lukas Bulwahn}
\maketitle

\begin{abstract}

This entry formalizes the two geometric theorems, Stewart's and Apollonius' theorem.
Stewart's Theorem~\cite{wikipedia:Stewart} relates the length of a triangle's cevian to the
lengths of the triangle's two sides. Apollonius' Theorem~\cite{wikipedia:Apollonius} is a
specialisation of Stewart's theorem, restricting the cevian to be the median. The proof
applies the law of cosines, some basic geometric facts about triangles and then simply transforms
the terms algebraically to yield the conjectured relation. The formalization in Isabelle can
closely follow the informal proofs described in the Wikipedia articles of those two theorems.

\end{abstract}

\tableofcontents

% sane default for proof documents
\parindent 0pt\parskip 0.5ex

% generated text of all theories
\input{session}

\nocite{*}

\bibliographystyle{abbrv}
\bibliography{root}

\end{document}

%%% Local Variables:
%%% mode: latex
%%% TeX-master: t
%%% End:
