\documentclass[11pt,a4paper]{article}
\usepackage{isabelle,isabellesym}

\usepackage{amssymb}
\usepackage{xspace}

% this should be the last package used
\usepackage{pdfsetup}

% urls in roman style, theory text in math-similar italics
\urlstyle{rm}
\isabellestyle{it}

\begin{document}

\title{An Incremental Simplex Algorithm with Unsatisfiable Core Generation\footnote{%
Supported by the Serbian Ministry of Education and Science grant 174021, 
by the SNF grant SCOPES IZ73Z0127979/1, and
by FWF (Austrian Science Fund) project Y757.
The authors are listed in alphabetical order regardless of individual
contributions or seniority.}}
\author{Filip Mari\'c \and Mirko Spasi\'c \and Ren\'e Thiemann}

\maketitle

\begin{abstract}
We present an Isabelle/HOL formalization and total correctness
proof for the incremental version of the Simplex algorithm which is
used in most state-of-the-art SMT solvers. It supports extraction of satisfying
assignments, extraction of minimal unsatisfiable cores, incremental assertion of constraints
and backtracking. The formalization relies on
stepwise program refinement, starting from a simple specification,
going through a number of refinement steps, and ending up in a fully
executable functional implementation. Symmetries present in the algorithm
are handled with special care. 
\end{abstract}


\tableofcontents

\section{Introduction}

This formalization closely follows the simplex algorithm as it is described by 
Dutertre and de~Moura~\cite{simplex-rad}. 

The original formalization has been developed and is extensively described 
by Spasi\'c and Mari\'c~\cite{SpasicMaric}.
It features a front-end that for a given set of constraints either returns
a satisfying assignment or the information that it is unsatisfiable.

The original formalization was extended by Thiemann in three different ways.
\begin{itemize}
\item The extended simplex method returns a minimal unsatisfiable core instead
  of just a bit ``unsatisfiable''.
\item The extension also contains an incremental interface to the simplex
  method where one can dynamically assert and retract linear constraints.
  In contrast, the original simplex formalization only offered an 
  interface which demands all constraints as input and which restarts
  the computation from scratch on every input.
\item The optimization of eliminating unused variables in the preprocessing 
  phase~\cite[Section~3]{simplex-rad} has been integrated in the formalization.
\end{itemize}
The first two of these extensions
required the introduction of \emph{indexed} constraints in combination with
generalised lemmas. In these generalisations, global constraints had to be replaced by arbitrary (indexed) subsets of constraints. 




\input{session}



\bibliographystyle{abbrv}
\bibliography{root}

\end{document}
