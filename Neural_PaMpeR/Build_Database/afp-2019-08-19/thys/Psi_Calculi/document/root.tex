\documentclass[11pt,a4paper]{article}
\usepackage{isabelle,isabellesym}
\usepackage{amssymb}
\usepackage[english]{babel}
\usepackage[latin1]{inputenc}

% this should be the last package used
\usepackage{pdfsetup}

% urls in roman style, theory text in math-similar italics
\urlstyle{rm}
\isabellestyle{it}


\begin{document}

\title{The pi-calculus}
\author{Jesper Bengtson}
\maketitle

\begin{abstract}

\end{abstract}

\tableofcontents

\section{Overview}

These theories formalise the following results for psi-calculi. Note
that there is only an early semantics for psi-calculi, although a late
one may appear later. 

\begin{itemize}
\item strong bisimilarity is preserved by all operators except the
  input-prefix 
\item strong equivalence is a congruence
\item weak bisimilarity is preserved by all operators except case and the
  input-prefix 
\item weak congruence is a congruence
\item strong equivalence respect the laws of structural congruence
\item all strongly equivalent agents are also weakly congruent which in
  turn are weakly bisimilar. Moreover, strongly equivalent agents are
  also strongly bisimilar 
\item as a corollary of the last two points, all mentioned equivalences
  respect the law of structural congruence 
\item for instances of psi-calculi where assertion composition satisfies
  weakening, the definition of weak bisimilarity can be simplified
  significantly and proven equivalent to the version that applies when
  weakening does not hold 
\item for certain versions of psi-calculi, sum can be encoded
\item for certain versions of psi-calculi, the tau-prefix can be encoded
  and when weakening is satisfied, all of the tau-laws hold. 
\end{itemize}

The file naming convention is hopefully self explanatory, where the
prefixes \emph{Strong} and \emph{Weak} denote that the file covers theories
required to formalise properties of strong and weak bisimilarity
respectively; files with the prefix \emph{Weaken} cover theories where
weakening holds for the static implication; if the file name contains
\emph{Sim} the theories cover simulation, file names containing \emph{Bisim}
cover bisimulation, and file names containing \emph{Cong} cover weak
congruence; files with the suffix \emph{Pres} deal with theories that
reason about preservation properties of operators such as a simulation
or bisimulation being preserved by a certain operator; files with the
suffix \emph{StructCong} reason about structural congruence.

For a complete exposition of all of theories, please consult Bengtson's
Ph. D. thesis \cite{bengtson:thesis}. A shorter presentation can be found in our TPHOLs paper
'Psi-calculi in Isabelle' from 2009 \cite{bengtson:tphols09}. There are also two LICS-papers that focus on the mathematical theories, rather than the Isabelle formalisations \cite{bengtson:lics09, johansson:lics10}.

% include generated text of all theories
\section{Formalisation}

\input{session}

\bibliographystyle{abbrv}
\bibliography{root}

\end{document}
