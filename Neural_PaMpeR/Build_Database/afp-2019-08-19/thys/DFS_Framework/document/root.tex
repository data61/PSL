\documentclass[11pt,a4paper]{report}
\usepackage{isabelle,isabellesym}
\usepackage[english]{babel}

% further packages required for unusual symbols (see also
% isabellesym.sty), use only when needed

%\usepackage{amssymb}
  %for \<leadsto>, \<box>, \<diamond>, \<sqsupset>, \<mho>, \<Join>,
  %\<lhd>, \<lesssim>, \<greatersim>, \<lessapprox>, \<greaterapprox>,
  %\<triangleq>, \<yen>, \<lozenge>

%\usepackage{eurosym}
  %for \<euro>

%\usepackage[only,bigsqcap]{stmaryrd}
  %for \<Sqinter>

%\usepackage{eufrak}
  %for \<AA> ... \<ZZ>, \<aa> ... \<zz> (also included in amssymb)

%\usepackage{textcomp}
  %for \<onequarter>, \<onehalf>, \<threequarters>, \<degree>, \<cent>,
  %\<currency>

% this should be the last package used
\usepackage{pdfsetup}

% urls in roman style, theory text in math-similar italics
\urlstyle{rm}
\isabellestyle{it}

% for uniform font size
%\renewcommand{\isastyle}{\isastyleminor}

\begin{document}

\title{A Framework for Verifying Depth-First Search Algorithms}
\author{Peter Lammich and Ren\'{e} Neumann}
\maketitle

\begin{abstract}
This entry presents a framework for the modular verification of DFS-based 
algorithms, which is described in our [CPP-2015] paper. 
It provides a generic DFS algorithm framework, that can be 
parameterized with user-defined actions on certain events 
(e.g. discovery of new node). 

It comes with an extensible library of invariants, which can be 
used to derive invariants of a specific parameterization.

Using refinement techniques, efficient implementations of the algorithms 
can easily be derived. 
Here, the framework comes with templates for
a recursive and a tail-recursive implementation, and also with several 
templates for implementing the data structures required by the DFS algorithm.

Finally, this entry contains a set of re-usable DFS-based algorithms, which 
illustrate the application of the framework.

\vfill
{\footnotesize
\begin{description}
\item[{[{CPP-2015}]}] Peter Lammich, Ren\'{e} Neumann: A Framework for Verifying Depth-First Search Algorithms. CPP 2015: 137-146
\end{description}
}
\end{abstract}

\clearpage

\tableofcontents

\clearpage

% sane default for proof documents
\parindent 0pt\parskip 0.5ex

% generated text of all theories
\input{session}

% optional bibliography
%\bibliographystyle{abbrv}
%\bibliography{root}

\end{document}

%%% Local Variables:
%%% mode: latex
%%% TeX-master: t
%%% End:
