\documentclass[11pt,a4paper]{article}
\usepackage{isabelle,isabellesym}
\usepackage{amsfonts, amsmath, amssymb}

% this should be the last package used
\usepackage{pdfsetup}

% urls in roman style, theory text in math-similar italics
\urlstyle{rm}
\isabellestyle{it}


\begin{document}

\title{The Hurwitz and Riemann $\zeta$ functions}
\author{Manuel Eberl}
\maketitle

\begin{abstract}
This entry builds upon the results about formal and analytic Dirichlet series to define the Hurwitz $\zeta$ function and,
based on that, the Riemann $\zeta$ function. This is done by first defining them for $\mathfrak{R}(z) > 1$ and then successively
extending the domain to the left using the Euler--MacLaurin formula.

Some basic results about these functions are also shown, such as their analyticity on $\mathbb{C}\setminus\{1\}$, that they have a simple pole with residue 1 at 1, their non-vanishing for $\mathfrak{R}(s)\geq 1$, their relation to the $\Gamma$ function, and the special values at negative integers and positive even integers -- including the famous $\zeta(-1) = -\frac{1}{12}$ and $\zeta(2) = \frac{\pi^2}{6}$.

Lastly, the entry also contains Euler's analytic proof of the infinitude of primes, based on the fact that $\zeta(s)$ has a pole at $s = 1$.
\end{abstract}

\tableofcontents
\newpage
\parindent 0pt\parskip 0.5ex

\input{session}

\bibliographystyle{abbrv}
\bibliography{root}

\end{document}

%%% Local Variables:
%%% mode: latex
%%% TeX-master: t
%%% End:
