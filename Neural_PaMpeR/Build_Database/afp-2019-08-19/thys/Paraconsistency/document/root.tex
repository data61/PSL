\documentclass[11pt,a4paper]{article}

\title{Paraconsistency}

\author{Anders Schlichtkrull \&\ J{\o}rgen Villadsen, DTU Compute, Denmark}

\date{\isadate\today}

\usepackage{datetime,isabelle,isabellesym,parskip,underscore}

\newdateformat{isadate}{\THEDAY\ \monthname[\THEMONTH] \THEYEAR}

\usepackage[cm]{fullpage}

\usepackage{pdfsetup}

\isabellestyle{tt}

\urlstyle{rm}

\renewcommand{\isachardoublequote}{}
\renewcommand{\isachardoublequoteopen}{}
\renewcommand{\isachardoublequoteclose}{}

\renewcommand{\isamarkupchapter}[1]{\clearpage\isamarkupsection{#1}}
\renewcommand{\isamarkupsection}[1]{\section*{#1}\addcontentsline{toc}{section}{#1}}
\renewcommand{\isamarkupsubsection}[1]{\medskip\subsection*{#1}}
\renewcommand{\isamarkupsubsubsection}[1]{\medskip\subsubsection*{#1}}

\renewcommand{\isabeginpar}{\par\ifisamarkup\relax\else\bigskip\fi}
\renewcommand{\isaendpar}{\par\bigskip}

\begin{document}

\makeatletter
\parbox[t]{\textwidth}{\centering\Huge\bfseries\@title}\par\kern5mm
\parbox[t]{\textwidth}{\centering\Large\bfseries\@author}\par\kern3mm
\parbox[t]{\textwidth}{\centering\bfseries\@date}\par\kern8mm
\makeatother

\begin{abstract}\normalsize\noindent
Paraconsistency is about handling inconsistency in a coherent way. In classical and intuitionistic
logic everything follows from an inconsistent theory. A paraconsistent logic avoids the explosion.
Quite a few applications in computer science and engineering are discussed in the Intelligent
Systems Reference Library Volume 110: Towards Paraconsistent Engineering (Springer 2016). We
formalize a paraconsistent many-valued logic that we motivated and described in a special issue on
logical approaches to paraconsistency (Journal of Applied Non-Classical Logics 2005). We limit
ourselves to the propositional fragment of the higher-order logic. The logic is based on so-called
key equalities and has a countably infinite number of truth values. We prove theorems in the logic
using the definition of validity. We verify truth tables and also counterexamples for non-theorems.
We prove meta-theorems about the logic and finally we investigate a case study.
\end{abstract}

\tableofcontents

\isamarkupsection{Preface}

The present formalization in Isabelle essentially follows our extended abstract \cite{Jensen+12}.
The Stanford Encyclopedia of Philosophy has a comprehensive overview of logical approaches to
paraconsistency \cite{Priest+15}. We have elsewhere explained the rationale for our paraconsistent
many-valued logic and considered applications in multi-agent systems and natural language semantics
\cite{Villadsen05-JANCL,Villadsen09,Villadsen10,Villadsen14}.

It is a revised and extended version of our formalization \url{https://github.com/logic-tools/mvl}
that accompany our chapter in a book on partiality published by Cambridge Scholars Press. The GitHub
link provides more information. We are grateful to the editors --- Henning Christiansen, M. Dolores
Jim\'{e}nez L\'{o}pez, Roussanka Loukanova and Larry Moss --- for the opportunity to contribute to
the book.

\input{session}

\clearpage\addcontentsline{toc}{section}{References}

\begin{thebibliography}{0}

\bibitem{Jensen+12}
A.~S. Jensen and J.~Villadsen.
\newblock
\emph{Paraconsistent Computational Logic}.
\newblock
In P.~Blackburn, K.~F.~J{\o}rgensen, N.~Jones, and E.~Palmgren, editors, 8th Scandinavian Logic
Symposium: Abstracts, pages 59--61, Roskilde University, 2012.

\bibitem{Priest+15}
G.~Priest, K.~Tanaka and Z.~Weber.
\newblock
\emph{Paraconsistent Logic}.
\newblock
In E.~N. Zalta et~al., editors, Stanford Encyclopedia of Philosophy, Online Entry
\url{http://plato.stanford.edu/entries/logic-paraconsistent/} Spring Edition, 2015.

\bibitem{Villadsen05-JANCL}
J.~Villadsen.
\newblock
\emph{Supra-logic: Using Transfinite Type Theory with Type Variables for Paraconsistency}.
\newblock
Logical Approaches to Paraconsistency, Journal of Applied Non-Classical Logics, 15(1):45--58, 2005.

\bibitem{Villadsen09}
J.~Villadsen.
\newblock
\emph{Infinite-Valued Propositional Type Theory for Semantics}.
\newblock
In J.-Y.~B\'{e}ziau and A.~Costa-Leite, editors, Dimensions of Logical Concepts, pages 277--297,
Unicamp Cole\c{c}.~CLE 54, 2009.

\bibitem{Villadsen10}
J.~Villadsen.
\newblock
\emph{Nabla: A Linguistic System Based on Type Theory}.
\newblock
Foundations of Communication and Cognition (New Series), LIT Verlag, 2010.

\bibitem{Villadsen14}
J.~Villadsen.
\newblock
\emph{Multi-dimensional Type Theory: Rules, Categories and Combinators for Syntax and Semantics}.
\newblock
In P.~Blache, H.~Christiansen, V.~Dahl, D.~Duchier, and J.~Villadsen, editors, Constraints and
Language, pages 167--189, Cambridge Scholars Press, 2014.

\end{thebibliography}

\end{document}
