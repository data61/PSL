\documentclass[11pt,a4paper]{article}

\usepackage{isabelle,isabellesym}
\usepackage{amssymb,ragged2e}
\usepackage{pdfsetup}

\isabellestyle{it}
\renewenvironment{isamarkuptext}{\par\isastyletext\begin{isapar}\justifying\color{blue}}{\end{isapar}}
\renewcommand\labelitemi{$*$}

\begin{document}

\title{Stone Algebras}
\author{Walter Guttmann}
\maketitle

\begin{abstract}
  A range of algebras between lattices and Boolean algebras generalise the notion of a complement.
  We develop a hierarchy of these pseudo-complemented algebras that includes Stone algebras.
  Independently of this theory we study filters based on partial orders.
  Both theories are combined to prove Chen and Gr\"atzer's construction theorem for Stone algebras.
  The latter involves extensive reasoning about algebraic structures in addition to reasoning in algebraic structures.
\end{abstract}

\tableofcontents

\section{Synopsis and Motivation}

This document describes the following four theory files:
\begin{itemize}
\item Lattice Basics is a small theory with basic definitions and facts extending Isabelle/HOL's lattice theory.
      It is used by the following theories.
\item Pseudocomplemented Algebras contains a hierarchy of algebraic structures between lattices and Boolean algebras.
      Many results of Boolean algebras can be derived from weaker axioms and are useful for more general models.
      In this theory we develop a number of algebraic structures with such weaker axioms.
      The theory has four parts.
      We first extend lattices and distributive lattices with a pseudocomplement operation to obtain (distributive) p-algebras.
      An additional axiom of the pseudocomplement operation yields Stone algebras.
      The third part studies a relative pseudocomplement operation which results in Heyting algebras and Brouwer algebras.
      We finally show that Boolean algebras instantiate all of the above structures.
\item Filters contains an order-/lattice-theoretic development of filters.
      We prove the ultrafilter lemma in a weak setting, several results about the lattice structure of filters and a few further results from the literature.
      Our selection is due to the requirements of the following theory.
\item Construction of Stone Algebras contains the representation of Stone algebras as triples and the corresponding isomorphisms \cite{ChenGraetzer1969,Katrinak1973}.
      It is also a case study of reasoning about algebraic structures.
      Every Stone algebra is isomorphic to a triple comprising a Boolean algebra, a distributive lattice with a greatest element, and a bounded lattice homomorphism from the Boolean algebra to filters of the distributive lattice.
      We carry out the involved constructions and explicitly state the functions defining the isomorphisms.
      A function lifting is used to work around the need for dependent types.
      We also construct an embedding of Stone algebras to inherit theorems using a technique of universal algebra.
\end{itemize}
Algebras with pseudocomplements in general, and Stone algebras in particular, appear widely in mathematical literature; for example, see \cite{BalbesDwinger1974,Birkhoff1967,Blyth2005,Graetzer1971}.
We apply Stone algebras to verify Prim's minimum spanning tree algorithm in Isabelle/HOL in \cite{Guttmann2016c}.

There are at least two Isabelle/HOL theories related to filters.
The theory \texttt{HOL/Algebra/Ideal.thy} defines ring-theoretic ideals in locales with a carrier set.
In the theory \texttt{HOL/Filter.thy} a filter is defined as a set of sets.
Filters based on orders and lattices abstract from the inner set structure; this approach is used in many texts such as \cite{BalbesDwinger1974,Birkhoff1967,Blyth2005,DaveyPriestley2002,Graetzer1971}.
Moreover, it is required for the construction theorem of Stone algebras, whence our theory implements filters this way.

Besides proving the results involved in the construction of Stone algebras, we study how to reason about algebraic structures defined as Isabelle/HOL classes without carrier sets.
The Isabelle/HOL theories \texttt{HOL/Algebra/*.thy} use locales with a carrier set, which facilitates reasoning about algebraic structures but requires assumptions involving the carrier set in many places.
Extensive libraries of algebraic structures based on classes without carrier sets have been developed and continue to be developed \cite{ArmstrongFosterStruthWeber2016,ArmstrongGomesStruth2016,ArmstrongGomesStruthWeber2016,DivasonAransay2016,FosterStruth2016,FurusawaStruth2016,GeorgescuLeusteanPreoteasa2016,GomesGuttmannHoefnerStruthWeber2016,GomesStruth2016,Guttmann2015a,KleinKolanskiBoyton2016,Preoteasa2016b,Preoteasa2016a,WamplerDoty2016}.
It is unlikely that these libraries will be converted to carrier-based theories and that carrier-free and carrier-based implementations will be consistently maintained and evolved; certainly this has not happened so far and initial experiments suggest potential drawbacks for proof automation \cite{FosterStruthWeber2011}.
An improvement of the situation seems to require some form of automation or system support that makes the difference irrelevant.

In the present development, we use classes without carrier sets to reason about algebraic structures.
To instantiate results derived in such classes, the algebras must be represented as Isabelle/HOL types.
This is possible to a certain extent, but causes a problem if the definition of the underlying set depends on parameters introduced in a locale; this would require dependent types.
For the construction theorem of Stone algebras we work around this restriction by a function lifting.
If the parameters are known, the functions can be specialised to obtain a simple (non-dependent) type that can instantiate classes.
For the construction theorem this specialisation can be done using an embedding.
The extent to which this approach can be generalised to other settings remains to be investigated.

\begin{flushleft}
\input{session}
\end{flushleft}

\bibliographystyle{abbrv}
\bibliography{root}

\end{document}

