\documentclass[8pt,a4paper]{article}
\usepackage[margin=2cm]{geometry}
\usepackage{isabelle,isabellesym}

% further packages required for unusual symbols (see also
% isabellesym.sty), use only when needed

%\usepackage{amssymb}
  %for \<leadsto>, \<box>, \<diamond>, \<sqsupset>, \<mho>, \<Join>,
  %\<lhd>, \<lesssim>, \<greatersim>, \<lessapprox>, \<greaterapprox>,
  %\<triangleq>, \<yen>, \<lozenge>

%\usepackage{eurosym}
  %for \<euro>

%\usepackage[only,bigsqcap]{stmaryrd}
  %for \<Sqinter>

%\usepackage{eufrak}
  %for \<AA> ... \<ZZ>, \<aa> ... \<zz> (also included in amssymb)

%\usepackage{textcomp}
  %for \<onequarter>, \<onehalf>, \<threequarters>, \<degree>, \<cent>,
  %\<currency>

% this should be the last package used
\usepackage{pdfsetup}

% urls in roman style, theory text in math-similar italics
\urlstyle{rm}
\isabellestyle{it}

% for uniform font size
%\renewcommand{\isastyle}{\isastyleminor}


\begin{document}

\title{Formalisation of an Adaptive State Counting Algorithm}
\author{Robert Sachtleben}
\maketitle


\begin{abstract}
    This entry provides a formalisation of a refinement of an adaptive state counting algorithm, used to test for reduction between finite state machines. The algorithm has been originally presented by Hierons in \cite{hierons} and was slightly refined by Sachtleben et al.\ in \cite{refinement}.
    Definitions for finite state machines and adaptive test cases are given and many useful theorems are derived from these.
    The algorithm is formalised using mutually recursive functions, for which it is proven that the generated test suite is sufficient to test for reduction against finite state machines of a certain fault domain.
    Additionally, the algorithm is specified in a simple WHILE-language and its correctness is shown using Hoare-logic.
\end{abstract}


\tableofcontents

% sane default for proof documents
\parindent 0pt\parskip 0.5ex

% generated text of all theories
\input{session}

% optional bibliography
\bibliographystyle{abbrv}
\bibliography{root}

\end{document}

%%% Local Variables:
%%% mode: latex
%%% TeX-master: t
%%% End:
