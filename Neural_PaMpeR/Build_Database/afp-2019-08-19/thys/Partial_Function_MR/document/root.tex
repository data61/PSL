\documentclass[11pt,a4paper]{article}
\usepackage{isabelle,isabellesym}

\usepackage{amsmath}

% this should be the last package used
\usepackage{pdfsetup}

% urls in roman style, theory text in math-similar italics
\urlstyle{rm}
\isabellestyle{it}

\newcommand\isakwd[1]{\textsf{\isa{#1}}}
\newcommand\isasimpmp{\isa{simplify-emp-main}}
\newcommand\parfun{\isakwd{partial-function}}
\newcommand\vect[1]{\overrightarrow{#1}}
\newcommand\fs{\isa{fs}}
\newcommand\xs{\isa{xs}}
\newcommand\xst{\isa{xs}_t}
\newcommand\inT{\isa{in}}
\newcommand\cprod[1]{({#1})}
\newcommand\outT{\isa{out}}
\newcommand\monad{\isa{monad}}
\newcommand\tto\Rightarrow
\newcommand\ar{\isa{ar}}
\newcommand\inj{\isa{inj}}
\newcommand\proj{\isa{proj}}
\newcommand\mapM{\isa{map-monad}}
\newcommand\curry{\isa{curry}}
\newcommand\case{\isakwd{case}}
\newcommand\of{\isakwd{of}}

\begin{document}

\title{Mutually Recursive Partial Functions\thanks{This research is supported by FWF (Austrian Science Fund) project P22767-N13. 
We thank Makarius Wenzel for several hints on how to properly localize our wrapper.}}
\author{Ren\'e Thiemann}
\maketitle

\begin{abstract}
  We provide a wrapper around the partial-function command which supports
  mutual recursion. 
  
  Our results have been used to simplify the development of mutually
  recursive parsers, e.g., a parser to convert external proofs given in XML 
  into some mutually recursive datatype within Isabelle/HOL.
\end{abstract}



\tableofcontents

\section{Introduction}

The partial function command of Krauss \cite{partial_function} turns monotone monadic function specifications
into equational theorems. Here, monadic means that the output type of the function must be a monad
like the option-monad. This is required to prohibit specifications like
\[
f\ x = 1 + f\ x
\]
which would immediately lead to a contradiction. Since the command produces unconditional 
equations, it is extremely helpful in writing possibly nonterminating functions which 
are amenable to code generation. For example, using \parfun, one can  write a 
recursive parser in Isabelle/HOL and can then use it in several target languages---without having to struggle
with a tedious termination proof which might have to reason about the internal 
state of the parser.

Unfortunately, the command currently does not support mutually recursive functions, which however
would be a convenient feature when writing parsers for mutually recursive datatypes.
To be more precise, a specification of a partial function has to be of the following shape
\begin{equation}
\label{nmr}
f\ \vect{\xs} = F\ f\ \vect{\xs}
\end{equation}
where $\vect{\xs}$ is a sequence of distinct variables and $F$ is an 
arbitrary monotone functional that may depend on $f$ and $\vect{\xs}$.

For mutually recursive functions we would like to specify functions in the more general form
\begin{align}
\notag f_1\ \vect{\xs_1} & = F_1\ \vect{\fs}\ \vect{\xs_1} \\
\label{mr} & \ \,\vdots \\
\notag f_n\ \vect{\xs_n} & = F_n\ \vect{\fs}\ \vect{\xs_n} 
\end{align}
where $\vect{\fs} = f_1, \ldots, f_n$ and $\vect{\xs_i}$ are the individual arguments to each of
the functions $f_i$.

In the following, we describe our wrapper around the partial function command which supports 
mutual recursion. We first 
synthesize a global function $g$ from the specifications in (\ref{mr}) which itself
has a defining equation in the form of (\ref{nmr}). 
Then we register $g$ and derive the defining equation for $g$ as theorem in Isabelle/HOL 
using $\parfun$. Afterwards,
it will be easy to define each $f_i$ in terms of $g$, and finally
derive the equations in (\ref{mr}) as theorems.

Let us now consider the details. Assume each $f_i$ has a type $\inT_{i,1} \tto
\ldots \tto \inT_{i,\ar(f_i)} \tto \outT_{f_i}\ \monad$, where for each $f$, $\ar(f)$ is the arity of $f$, and $\monad$ is the common monad.
For $g$ there will only be one input, and this input has type $\cprod{\inT_{f_1}} + \ldots + \cprod{\inT_{f_n}}$:
each sequence of input types
$\inT_{i,1}, \ldots, \inT_{i,\ar(f_i)}$ is first transformed into a single argument of 
type $\cprod{\inT_{f_i}} := \inT_{i,1} \times \ldots \times \inT_{i,\ar(f_i)}$,
and afterwards the sum type is used to distinguish between the inputs of the individual functions.
Similarly, the output type of $g$ will be $(\outT_{f_1} + \ldots + \outT_{f_n})\ \monad$.
Note that we did not choose ${\outT_{f_1}\ \monad} + \ldots + {\outT_{f_n} \monad}$ as output of $g$ 
as it is not monadic, and thus, $g$ would not be definable via \parfun.

Next, we define $g$ via a single equation which can then be passed to \parfun. Here, 
we have to 
\begin{itemize}
\item convert between tuples and sequences of arguments via currying and uncurrying. 
  To this end, we use the predefined \curry-function for currying and for uncurrying we
  perform pattern matching in expressions like $\lambda \cprod{\xs}. h\ \vect{xs}$ which
  take a tuple of variables as argument and then feed these variables sequentially to 
  some function $h$.
\item convert between argument and sum-types. To this end, we use
  constructors $\inj_i$ of type $\alpha_i \tto \alpha_1 + \ldots + \alpha_i + \ldots \alpha_n$,
  and destructors $\proj_i$ which work in exactly the opposite direction. Moreover, we perform
  case-analyses via pattern matching on the $\inj_i$'s. Note that internally each $\inj_i$ is encoded
  via repeated usage of the constructors $\isa{Inl}$ and $\isa{Inr}$ of Isabelle/HOL's 
  \isa{sum}-type, and similarly we
  nest \isa{Projl} and \isa{Projr} to encode arbitrary $\proj_i$-functions.
\item work within the monad to combine the various result types into a single one. 
  To this end,
  we demand that there is some $\mapM$-function which lifts an operation $\alpha \tto \beta$
  to a function of type $\alpha\ \monad \tto \beta\ \monad$. In general, these mappings may
  also take several functions as input, depending on the number of type-variables of the 
  monad-constructor. For each kind of monad that should be supported by our method,
  a user-defined $\mapM$ function can be registered. It is important, to also register a
  monotonicity lemma of each $\mapM$ function within the partial function package. Otherwise,
  monotonicity proofs for $g$ will most likely fail.
\end{itemize}
Putting everything together, we setup the following equation
\begin{align}
  g\ x = \case\ x\ \of \notag\\
    \inj_1 \xst & \tto \mapM\ \inj_1\ ((\lambda \cprod{\xs}. F_1\ \vect{\fs'}\ \vect\xs)\ \xst) \notag\\
   \mid \ldots\hspace{1.8em} \label{gsimp} \\
   \mid \inj_n \xst & \tto \mapM\ \inj_n\ ((\lambda \cprod{\xs}. F_n\ \vect{\fs'}\ \vect\xs)\ \xst)\notag
\end{align}
where $\vect{\fs'}$ is the sequence of abbreviations $f'_1$, \ldots, $f'_n$ and where
\begin{equation}
\label{fi'}
f'_i = \curry\ (\lambda \xst. \ \mapM\ \proj_i\ (g\ (\inj_i\ \xst)))
\end{equation}
Once, $g$ has been defined using \parfun, we obtain Equality (\ref{gsimp}) as a theorem.
Afterwards, it is easy to define
\begin{equation}
\label{fi}
f_i = \curry\ (\lambda \xst.\ \mapM\ \proj_i\ (g\ (\inj_i\ \xst)))
\end{equation}
and it remains to derive the equations in (\ref{mr}) as theorems.
To this end, first note the difference in (\ref{fi'}) and (\ref{fi}). In the former, $g$ is a free
variable which should be defined as a constant at that point, 
whereas $g$ is already the newly defined constant in (\ref{fi}). Obviously, at this point one
can now replace the abbreviations (\ref{fi'}) in Equation (\ref{gsimp}) by the real constants $f_i$ via
the defining equations (\ref{fi}). This yields the following modified theorem for $g$ 
where now $\vect{\fs}$ is the sequence $f_1,\dots,f_n$.
\begin{align}
  g\ x = \case\ x\ \of \notag\\
    \inj_1 \xst & \tto \mapM\ \inj_1\ ((\lambda \cprod{\xs}. F_1\ \vect{\fs}\ \vect\xs)\ \xst) \notag\\
   \mid \ldots\hspace{1.8em} \label{gsimp2} \\
   \mid \inj_n \xst & \tto \mapM\ \inj_n\ ((\lambda \cprod{\xs}. F_n\ \vect{\fs}\ \vect\xs)\ \xst)\notag
\end{align}
Now it is indeed easy to derive the desired equations in (\ref{mr}):
\begin{align*}
f_i\ \vect\xs & \stackrel{(\ref{fi})}= (\curry\ (\lambda \xst.\ \mapM\ \proj_i\ (g\ (\inj_i\ \xst)))) \ \vect\xs \\
& \stackrel{(\star)}= \mapM\ \proj_i\ (g\ (\inj_i\ (\vect\xs)))) \\
& \stackrel{(\ref{gsimp2})}= \mapM\ \proj_i\ (\mapM\ \inj_i\ (F_i\ \vect{\fs}\ \vect\xs))\\
& \stackrel{(\star\star)}= F_i\ \vect{\fs}\ \vect\xs
\end{align*}
Here, ($\star$) used the definition of \curry\ and splitting of tuples, and for $(\star\star)$
we demand that $\mapM$ is compositional and that $\mapM$ applied on the identity function is the
identity function itself.


% include generated text of all theories
\section{Implementation}

\subsection{Known limitations}
\begin{itemize}
\item The method does only provide equational theorems. It does not convert
  the induction rule for the global function $g$ from the partial function command
  into an induction rule for the set of mutually recursive functions.
\end{itemize}
 

\input{session}

\bibliographystyle{abbrv}
\bibliography{root}

\end{document}
