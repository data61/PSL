\documentclass[11pt,a4paper]{article}
\usepackage{isabelle,isabellesym}

% further packages required for unusual symbols (see also
% isabellesym.sty), use only when needed

\usepackage{amssymb}
  %for \<leadsto>, \<box>, \<diamond>, \<sqsupset>, \<mho>, \<Join>,
  %\<lhd>, \<lesssim>, \<greatersim>, \<lessapprox>, \<greaterapprox>,
  %\<triangleq>, \<yen>, \<lozenge>


% this should be the last package used
\usepackage{pdfsetup}



% urls in roman style, theory text in math-similar italics
\urlstyle{rm}
\isabellestyle{it}

% for uniform font size
%\renewcommand{\isastyle}{\isastyleminor}


\begin{document}

\title{Routing}
\author{Julius Michaelis, Cornelius Diekmann}
\maketitle

\begin{abstract}  
	This entry contains definitions for routing with routing tables/longest prefix matching.

	A routing table entry is modelled as a record of a prefix match, a metric, an output port, and an optional next hop.
	A routing table is a list of entries, sorted by prefix length and metric.
	Additionally, a parser and serializer for the output of the ip-route command, a function to create a relation from output port to corresponding destination IP space, and a model of a linux style router are included.
\end{abstract}

\tableofcontents

% sane default for proof documents
\parindent 0pt\parskip 0.5ex

% generated text of all theories
\input{session}

% optional bibliography
%\bibliographystyle{abbrv}
%\bibliography{root}

\end{document}

%%% Local Variables:
%%% mode: latex
%%% TeX-master: t
%%% End:
