\documentclass[11pt,a4paper]{article}
\usepackage{isabelle,isabellesym}

\usepackage{amssymb}
\usepackage{xspace}

% this should be the last package used
\usepackage{pdfsetup}

% urls in roman style, theory text in math-similar italics
\urlstyle{rm}
\isabellestyle{it}

\newcommand\isafor{\textsf{IsaFoR}}
\newcommand\ceta{\textsf{Ce\kern-.18emT\kern-.18emA}}
\newcommand\rats{\mathbb{Q}}
\newcommand\ints{\mathbb{Z}}
\newcommand\reals{\mathbb{R}}
\newcommand\complex{\mathbb{C}}

\newcommand\rai{real algebraic number\xspace}
\newcommand\rais{real algebraic numbers\xspace}

\begin{document}

\title{Subresultants\footnote{Supported by FWF (Austrian Science Fund) project Y757.}}
\author{Sebastiaan Joosten, Ren\'e Thiemann and Akihisa Yamada}
\maketitle

\begin{abstract}
We formalize the theory of subresultants and the subresultant polynomial
remainder sequence as described by Brown and Traub. 
As a result, we obtain efficient certified algorithms for computing the resultant
and the greatest common divisor of polynomials. 
\end{abstract}

\tableofcontents

\section{Introduction}

Computing the gcd of two polynomials can be done via the Euclidean algorithm, 
if the domain of the polynomials is a field. For non-field polynomials, one has to
replace the modulo operation by the pseudo-modulo operation, which results in
the exponential growth of coefficients in the gcd algorithm.
To counter this problem, one may divide the 
intermediate polynomials by their contents in every iteration of the gcd algorithm. 
This is precisely the way how currently resultants and gcds are computed in Isabelle.

Computing contents in every iteration is a costly operation, and therefore Brown and
Traub have developed the subresultant PRS (polynomial remainder sequence) algorithm \cite{Brown,BrownTraub}. 
It avoids intermediate content computation and at the same time keeps the coefficients small, i.e., the coefficients grow at most polynomially.

The soundness of the subresultant PRS gcd algorithm is in principle similar to the Euclidean algorithm, i.e., 
the intermediate polynomials that are computed in both algorithms differ only by a constant factor.
The major problem is to prove that all the performed divisions are indeed exact divisions.
To this end, we formalize the fundamental theorem of Brown and Traub as well as the resulting algorithms 
by following the original (condensed) proofs.
This is in contrast to a similar Coq formalization by Mahboubi \cite{Mahboubi06}, which follows another proof 
based on polynomial determinants.

As a consequence of the new algorithms, we significantly increased the speed of the algebraic number implementation \cite{AlgebraicNumbers} which
heavily relies upon the computation of resultants of bivariate polynomials. 

% include generated text of all theories
\input{session}



\bibliographystyle{abbrv}
\bibliography{root}

\end{document}
