\documentclass[11pt,a4paper]{report}
\usepackage{latexsym}
\usepackage{amsmath}
\usepackage{amssymb}
\usepackage[english]{babel}
\usepackage[utf8]{inputenc}
\usepackage[only,bigsqcap]{stmaryrd}
\usepackage{wasysym}
\usepackage{eufrak}
\usepackage{textcomp}
%\usepackage{apalike}
%\usepackage{times}

\usepackage{isabelle,isabellesym} 

% this should be the last package used
\usepackage{pdfsetup}

% proper setup for best-style documents
\urlstyle{rm}   
\isabellestyle{it}

\hyphenation{Isabelle}

\date{15th May 2003}
\parindent 0pt\parskip 0.5ex

\begin{document}


\title{Formalizing Integration Theory, with an Application to
  Probabilistic Algorithms}


\author{Stefan Richter\\
  LuFG Theoretische Informatik\\
  RWTH Aachen\\
  Ahornstraße 55\\
  52056 Aachen\\
  FRG\\
  \url{richter@informatik.rwth-aachen.de}}

\date{\today}
      
\maketitle

\pagestyle{headings}
\tableofcontents

% include generated text of all theories
% \input{session}
\newpage
\pagestyle{headings}
\chapter{Introduction}
  This development provides an efficient, extensible, machine checked collections framework for use
  in Isabelle/HOL. The library adopts the concepts of interface, implementation and generic algorithm
  known from object oriented (collection) libraries like the C++ Standard Template Library\cite{C++STL} or 
  the Java Collections Framework\cite{JavaCollFr} and makes them available in the Isabelle/HOL environment.

  The library uses data refinement techniques to refine an abstract specification (in terms of high-level concepts such as sets) to a more concrete implementation (based on collection datastructures like red-black-trees).
  This allows algorithms to be proven on the abstract level at which proofs are simpler because they are not cluttered with low-level details.

  The code-generator of Isabelle/HOL can be used to generate efficient code in all supported target languages, i.e. Haskell, SML, and OCaml.

  For more documentation and introductory material refer to the userguide (Section~\ref{thy:Userguide}) and the ITP-2010 paper \cite{LammichLochbihler2010ITP}.

\input{Sigma_Algebra}  
\input{MonConv}
\input{Measure}
\newpage
\input{RealRandVar}
\chapter{Integration}
\label{cha:integration}

The chapter at hand assumes a central position in the present paper. The Lebesgue
integral is defined and its characteristics are shown in
\ref{sec:stepwise-approach}. To illustrate the problems arising in
doing so, we first look at implementation alternatives that did not
work out. 

\input{Failure}
\input{Integral}
\chapter{Epilogue}
\label{cha:epilogue}

To come to a conclusion, a few words shall
subsume the work done and point out opportunities for future research at the same time.

What has been achieved? After opening with some
introductory notes, we began translating the language of measure
theory into machine checkable text. For the material in
section \ref{sec:preliminaries}, this had been done before. Besides laying the
foundation for the development, the style of presentation should make it
noteworthy.  

It is a particularity of the present work that its theories are
written in the Isar language, a declarative proof language that aims
to be ``intelligible''. This is not a novelty, nor is it the author's
merit. Still, giving full formal proofs in a text intended to be read
by people is in a way experimental. Clearly, it is bound to put some
strain on the reader. Nevertheless, I hope that we have made a little
step towards formalizing mathematical knowledge in a way that is
equally suitable for computation and understanding. One aim of the
research done has been to demonstrate the viability of this approach.
Unquestionably, there is plenty room for improvement regarding the
quality of presentation. The language itself has, in my opinion,
proven to be fit for a wide range of applications, including the
classical mathematics we used it for.

Returning to a more content-centered viewpoint, we discussed the
measurability of real-valued functions in section \ref{sec:realrandvar}. As
explained there, earlier scholarship has resulted in related theories
for the MIZAR environment though the development seems to have
stopped. Anyway, the mathematics covered should be new to HOL-based
systems. 

More functions could obviously be demonstrated to be random
variables. We shortly commented on an alternative approach in the
section just mentioned. It is applicable to continuous functions,
proving these measurable all at once. Efforts on topological spaces
would be required, but they constitute an interesting field
themselves, so it is probably worth the while.

In the third chapter, integration in the Lebesgue style
has been looked at in depth. To my knowledge, no similar theory had
been developed in a theorem prover up to this point.  We managed to
systematically establish the integral of increasingly complex
functions. Simple or nonnegative functions ought to be treated in
sufficient detail by now. Of course, the repository of potential
supplementary facts is vast. Convergence theorems, as well as the
interrelationship with differentiation or concurrent integral concepts,
are but a few examples. They leave ample space for subsequent work.

A shortcoming of the present development lies in the lack of
user assistance. Greater care could be taken to ensure automatic
application of appropriate simplification rules --- or to design such
rules in the first place. Likewise, the principal requirement of
integrability might hinder easy usage of the integral. Fixing a
default value for undefined integrals could possibly make some case
distinctions obsolete. Facets like these have not been addressed in
their due extent.

 
\begin{flushleft}
\bibliographystyle{plain}
\bibliography{root}
\end{flushleft}

\end{document}
