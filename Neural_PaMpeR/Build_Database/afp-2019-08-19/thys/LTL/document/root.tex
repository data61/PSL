\documentclass[11pt,a4paper]{article}

\usepackage[english]{babel}
\usepackage[utf8]{inputenc}

\usepackage{isabelle,isabellesym}
\usepackage{amssymb}
\usepackage[only,bigsqcap]{stmaryrd}

\usepackage[T1]{fontenc}

% this should be the last package used
\usepackage{pdfsetup}

% urls in roman style, theory text in math-similar italics
\urlstyle{rm}
\isabellestyle{it}

% for uniform font size
\renewcommand{\isastyle}{\isastyleminor}

\begin{document}

\title{Linear Temporal Logic}
\author{Salomon Sickert}
\maketitle

\begin{abstract}
This theory provides a formalisation of linear temporal logic (LTL) and unifies previous formalisations within the AFP. This entry establishes syntax and semantics for this logic and decouples it from existing  entries, yielding a common environment for theories reasoning about LTL. Furthermore a parser written in SML and an executable simplifier are provided.
\end{abstract}

\tableofcontents

% sane default for proof documents
\parindent 0pt\parskip 0.5ex

% generated text of all theories
\input{session}

\bibliographystyle{abbrv}
\bibliography{root}

\end{document}

%%% Local Variables:
%%% mode: latex
%%% TeX-master: t
%%% End:
