\documentclass[11pt,a4paper]{article}
\usepackage{isabelle,isabellesym,latexsym}

% further packages required for unusual symbols (see also
% isabellesym.sty), use only when needed

\usepackage{amssymb}
  %for \<leadsto>, \<box>, \<diamond>, \<sqsupset>, \<mho>, \<Join>,
  %\<lhd>, \<lesssim>, \<greatersim>, \<lessapprox>, \<greaterapprox>,
  %\<triangleq>, \<yen>, \<lozenge>

%\usepackage{eurosym}
  %for \<euro>

%\usepackage[only,bigsqcap]{stmaryrd}
  %for \<Sqinter>

%\usepackage{eufrak}
  %for \<AA> ... \<ZZ>, \<aa> ... \<zz> (also included in amssymb)

%\usepackage{textcomp}
  %for \<onequarter>, \<onehalf>, \<threequarters>, \<degree>, \<cent>,
  %\<currency>

% this should be the last package used
\usepackage{pdfsetup}

% urls in roman style, theory text in math-similar italics
\urlstyle{rm}
\isabellestyle{it}

% for uniform font size
%\renewcommand{\isastyle}{\isastyleminor}


\begin{document}

\title{Gr\"obner Bases Theory}
\author{Fabian Immler and Alexander Maletzky\thanks{Supported by the Austrian 
Science Fund (FWF): grant no. W1214-N15 (project DK1) and grant no. P 29498-N31}}
\maketitle

\begin{abstract}
This formalization is concerned with the theory of Gr\"obner bases in 
(commutative) multivariate polynomial rings over fields, originally developed 
by Buchberger in his 1965 PhD thesis. Apart from the statement and proof of the 
main theorem of the theory, the formalization also implements algorithms for actually computing 
Gr\"obner bases, thus allowing to effectively decide ideal membership in finitely generated 
polynomial ideals. Furthermore, all functions can be executed on a concrete 
representation of multivariate polynomials as association lists.
\end{abstract}

\tableofcontents

% sane default for proof documents
\parindent 0pt\parskip 0.5ex

\newpage
\section{Introduction}

The theory of Gr\"obner bases, invented by Buchberger in~\cite{Buchberger1965,Buchberger1970}, is 
ubiquitous in many areas of 
computer algebra and beyond, as it allows to effectively solve a multitude of 
interesting, non-trivial problems of polynomial ideal theory. Since its 
invention in the mid-sixties, the theory has already seen a whole range of 
extensions and generalizations, some of which are present in this formalization:
\begin{itemize}
 \item Following~\cite{Kreuzer2000}, the theory is formulated for vector-polynomials instead of 
ordinary scalar polynomials, thus allowing to compute Gr\"obner bases of syzygy modules.
 
 \item Besides Buchberger's original algorithm, the formalization also features Faug\`ere's $F_4$ 
algorithm~\cite{Faugere1999} for computing Gr\"obner bases.

 \item All algorithms for computing Gr\"obner bases incorporate criteria to avoid useless pairs; 
see~\cite{Buchberger1979} for details.

 \item Reduced Gr\"obner bases have been formalized and can be computed by a formally verified 
algorithm, too.
\end{itemize}

For further information about Gr\"obner bases theory the interested reader may consult the 
introductory paper~\cite{Buchberger1998a} or literally any book on 
commutative/computer algebra, e.\,g.~\cite{Adams1994,Kreuzer2000}.

\subsection{Related Work}

The theory of Gr\"obner bases has already been formalized in a couple of other 
proof assistants, listed below in alphabetical order:
\begin{itemize}
 \item ACL2~\cite{Medina-Bulo2010},
 \item Coq~\cite{Thery2001,Jorge2009},
 \item Mizar~\cite{Schwarzweller2006}, and
 \item Theorema~\cite{Buchberger2003,Maletzky2016b}.
\end{itemize}

Please note that this formalization must not be confused with the 
\textit{algebra} proof method based on Gr\"obner bases~\cite{Chaieb2007}, which 
is a completely independent piece of work: our results could in principle be 
used to formally prove the correctness and, to some extent, completeness of 
said proof method.

\subsection{Future Work}

This formalization can be extended in several ways:
\begin{itemize}
 \item One could formalize signature-based algorithms for computing Gr\"obner bases, as for 
instance Faug\`ere's $F_5$ algorithm~\cite{Faugere2002}. Such algorithms are typically more 
efficient than Buchberger's algorithm.

 \item One could establish the connection to \emph{elimination theory}, 
exploiting the well-known \emph{elimination property} of Gr\"obner bases 
w.\,r.\,t. certain term-orders (e.\,g. the purely lexicographic one). This 
would enable the effective simplification (and even solution, in some sense) of
systems of algebraic equations.

 \item One could generalize the theory further to cover also \emph{non-commutative} Gr\"obner 
bases~\cite{Mora1994}.
\end{itemize}

% generated text of all theories
\input{session}

% optional bibliography
\bibliographystyle{abbrv}
\bibliography{root}

\end{document}

%%% Local Variables:
%%% mode: latex
%%% TeX-master: t
%%% End:
