\documentclass[11pt,a4paper]{article}
\usepackage{isabelle,isabellesym}
\usepackage{amssymb}
\usepackage{pdfsetup}

\urlstyle{rm}
\isabellestyle{it}

\begin{document}

\title{Von Neumann Morgenstern Utility Theorem \thanks{%
This work is supported by the Austrian Science Fund (FWF) project P26201 and the European Research Council (ERC) grant no 714034 \emph{SMART}.}}

\author{Julian Parsert \and Cezary Kaliszyk}

\maketitle

\begin{abstract}
Utility functions form an essential part of game theory and
economics. In order to guarantee the existence of utility
functions most of the time sufficient properties are assumed in an axiomatic manner.
One famous and very common set of such assumptions is that of expected utility theory.
Here, the rationality, continuity, and independence of preferences is assumed.
The von-Neumann-Morgenstern Utility theorem shows that these assumptions are necessary
and sufficient for an expected utility function to exists.
This theorem was proven by Neumann and Morgenstern in ``Theory of Games and Economic Behavior''
which is regarded as one of the most influential works in game theory.

We formalize these results in Isabelle/HOL. The formalization includes
formal definitions of the underlying concepts including continuity
and independence of preferences.
\end{abstract}

\tableofcontents

\parindent 0pt\parskip 0.5ex
\input{session}

\section{Related work}
Formalizations in Social choice theory has been formalized by
Wiedijk~\cite{Wiedijk2009},
Nipkow~\cite{DBLP:journals/afp/Nipkow08b}, and Gammie~\cite{SenSocialChoice:AFP,StableMatching:AFP}.
Vestergaard~\cite{DBLP:journals/ipl/Vestergaard06},
Le Roux, Martin-Dorel,
and Soloviev~\cite{DBLP:conf/tphol/Roux09,DBLP:journals/corr/abs-1709-02096} provide formalizations
of results in game theory.
A library for algorithmic game theory in Coq is described in\cite{JFR7235}.

Related work in economics includes the verification of financial
systems~\cite{passmoreInf}, binomial pricing
models~\cite{DBLP:conf/cade/EchenimP17},
and VCG-Auctions~\cite{kerber2013developing}. In
microeconomics we discussed a formalization of two economic models and
the First Welfare Theorem~\cite{Parsert:2018:FMF:3176245.3167100}.

To our knowledge the only work that uses expected utility theory is
that of Eberl~\cite{Randomised:Social:ChoiceAFP}.
Since we focus on the underlying theory of expected utility, we found that
there is only little overlap.

\bibliographystyle{abbrv}
\bibliography{root}

\end{document}
