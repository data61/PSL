\documentclass[11pt,DIV10,a4paper,twoside=semi,openright,titlepage]{scrreprt}
\usepackage{fixltx2e}
%%%%%%%%%%%%%%%%%%%%%%%%%%%%%%%%%%%%%%%%%%%%%%%%%%%%%%%%%%%%%%%%%%%%%%%
%%% Overrides the (rightfully issued) warning by Koma Script that \rm
%%% etc. should not be used (they are deprecated since more than a
%%% decade)
  \DeclareOldFontCommand{\rm}{\normalfont\rmfamily}{\mathrm}
  \DeclareOldFontCommand{\sf}{\normalfont\sffamily}{\mathsf}
  \DeclareOldFontCommand{\tt}{\normalfont\ttfamily}{\mathtt}
  \DeclareOldFontCommand{\bf}{\normalfont\bfseries}{\mathbf}
  \DeclareOldFontCommand{\it}{\normalfont\itshape}{\mathit}
%%%%%%%%%%%%%%%%%%%%%%%%%%%%%%%%%%%%%%%%%%%%%%%%%%%%%%%%%%%%%%%%%%%%%%%

\usepackage{isabelle,isabellesym}
\usepackage{stmaryrd}
\usepackage{paralist}
\usepackage{xspace}
\usepackage{amsmath}
\usepackage[USenglish]{babel}
\newcommand{\testgen}{HOL-TestGen\xspace}
\newcommand{\testgenFW}{HOL-TestGen/FW\xspace}
\usepackage[numbers, sort&compress, sectionbib]{natbib}
\usepackage{graphicx}
\usepackage{color}
\sloppy

\usepackage{amssymb}
\newcommand{\isadefinition} {{\operatorname{definition}}}
\newcommand{\types} {{\operatorname{type\_synonym}}}
\newcommand{\datatype} {{\operatorname{datatype}}}
\newcommand{\ap}{\,}


\newcommand{\dom}{\mathrm{dom}}
\newcommand{\ran}{\mathrm{ran}}
\newcommand{\ofType}{\!::\!}
\newcommand{\HolBin}[0]{\ensuremath{\mathrm{bin}}}
\newcommand{\HolNum}[0]{\ensuremath{\mathrm{num}}}
\newcommand{\HolBoolean}[0]{\ensuremath{\mathrm{bool}}}
\newcommand{\HolString}[0]{\ensuremath{\mathrm{string}}}
\newcommand{\HolInteger}[0]{\ensuremath{\mathrm{int}}}
\newcommand{\HolNat}[0]{\ensuremath{\mathrm{nat}}}
\newcommand{\HolReal}[0]{\ensuremath{\mathrm{real}}}
\newcommand{\HolSet}[1]{#1\ap\ensuremath{\mathrm{set}}}
\newcommand{\HolList}[1]{#1\ap\ensuremath{\mathrm{list}}}
%\newcommand{\HolOrderedSet}[1]{#1~\ensuremath{\mathrm{orderedset}}}
\newcommand{\HolMultiset}[1]{#1\ap\ensuremath{\mathrm{multiset}}}
\newcommand{\classType}[2]{#1\ap\ensuremath{\mathrm{#2}}}
\newcommand{\bottom}{\bot}
\DeclareMathOperator{\HolSome}{Some}
\DeclareMathOperator{\HolNone}{None}
\DeclareMathOperator{\Poverride}{\oplus}
\DeclareMathOperator{\prodTwo}{\otimes_2}

\newcommand{\HolMkSet}[1]{\operatorname{set} #1}
\newcommand{\spot}{.\;}
\newcommand{\where} {{\operatorname{where}}}
\DeclareMathOperator{\HolIf}{if}
\DeclareMathOperator{\HolLet}{let}
\DeclareMathOperator{\HolIn}{in}
\DeclareMathOperator{\HolThen}{then}
\DeclareMathOperator{\HolElse}{else}


\newcommand{\isasymmodels}{\isamath{\models}}
\newcommand{\HOL}{HOL}

\newcommand{\ie}{i.\,e.}
\newcommand{\eg}{e.\,g.}

\usepackage{pdfsetup}

\urlstyle{rm}
\isabellestyle{it}
\renewcommand{\isastyle}{\isastyleminor}

\pagestyle{empty} 
\begin{document}
\renewcommand{\subsubsectionautorefname}{Section}
\renewcommand{\subsectionautorefname}{Section}
\renewcommand{\sectionautorefname}{Section}
\renewcommand{\chapterautorefname}{Chapter}
\newcommand{\subtableautorefname}{\tableautorefname}
\newcommand{\subfigureautorefname}{\figureautorefname}

\title{Formal Network Models and Their Application to Firewall Policies\\ (UPF-Firewall)}
\author{Achim D. Brucker\footnotemark[1] \quad
        Lukas Br\"ugger\footnotemark[2]  \quad
        Burkhart Wolff\footnotemark[3]\\[1.5em]
  \normalsize
  \normalsize\footnotemark[1]~Department of Computer Science, The University of Sheffield, Sheffield, UK
  \texorpdfstring{\\}{}
  \normalsize\href{mailto:"Achim D. Brucker"
    <a.brucker@sheffield.ac.uk>}{a.brucker@sheffield.ac.uk}\\[1em]
  %
  \normalsize\footnotemark[2]Information Security, ETH Zurich, 8092 Zurich, Switzerland
  \texorpdfstring{\\}{}
  \normalsize\href{mailto:"Lukas Bruegger"
    <lukas.a.bruegger@gmail.com>}{Lukas.A.Bruegger@gmail.com}\\[1em]
  %
  \normalsize\footnotemark[3]~Univ. Paris-Sud, Laboratoire LRI,
  UMR8623, 91405 Orsay, France
  France\texorpdfstring{\\}{}
  \normalsize\href{mailto:"Burkhart Wolff" <burkhart.wolff@lri.fr>}{burkhart.wolff@lri.fr}
}

\pagestyle{empty}
\publishers{%
  \normalfont\normalsize%
    \centerline{\textsf{\textbf{\large Abstract}}}
    \vspace{1ex}%
    \parbox{0.8\linewidth}{%
      We present a formal model of network protocols and their
      application to modeling firewall policies. The formalization is
      based on the \emph{Unified Policy Framework} (UPF). The
      formalization was originally developed with for generating test
      cases for testing the security configuration actual firewall and
      router (middle-boxes) using HOL-TestGen. Our work focuses on
      modeling application level protocols on top of tcp/ip. 
    }
}

\maketitle
\cleardoublepage
\pagestyle{plain}
\tableofcontents
\cleardoublepage

\chapter{Introduction}
  \section{Introduction}

Linearizability~\cite{HerlihyWing90Linearizability} is a key design methodology
for reasoning about implementations of concurrent abstract data types in both
shared memory and message passing systems. It presents the illusion that
operations execute sequentially and fault-free, despite the asynchrony and
faults that are often present in a concurrent system, especially a distributed
one.

However, devising complete linearizable objects is very difficult, especially
in the presence of process crashes and asynchrony, requiring complex algorithms
(such as Paxos~\cite{Lamport98PartTimeParliament}) to work correctly under
general circumstances, and often resulting in bad average-case behavior.
Concurrent algorithm designers therefore resort to speculation, i.e.\ to
optimizing existing algorithms to handle common scenarios more efficiently.
More precisely, a speculative systems has a fall-back mode that works in all
situations and several optimization modes, each of which is very efficient in
a particular situation but might not work at all in some other situation. By
observing its execution, a speculative system speculates about which particular
situation it will be subject to and chooses the most efficient mode for that
situation. If speculation reveals wrong, a new speculation is made in light of
newly available observations. Unfortunately, building speculative system ad-hoc
results in protocols so complex that it is no longer tractable to prove their
correctness.

We the specification of the SLin (a shorthand for Speculative
Linearizabibility) I/O-automaton~\cite{Lynch89anintroduction}, which can
be used to build a speculatively linearizable algorithm out of independent
modules that each implement the different modes of the speculative
algorithm. The SLin I/O-automaton is at the heart of the Speculative
Linearizability framework~\cite{Losa2014,GKL2012SpeculativeLinearizability}.
The Speculative Linearizability framework first appeared
in~\cite{GKL2012SpeculativeLinearizability} and was later refined
in~\cite{Losa2014}. This development is based on the later~\cite{Losa2014}.

The SLin I/O-automaton produces traces that are linearizable with respect to
a generic type of object. Moreover, the composition of two instances of the
SLin I/O-automaton behaves like a single instance. Hence it is guaranteed
that the composition of any number of instances of the SLin I/O-automaton is
linearizable. In this formal development, we prove the idempotence theorem,
i.e.\ that the composition of two instances of the SLin I/O-automaton is itself
an implementation of the SLin I/O-automaton.

The properties stated above simplify the development and analysis of speculative
systems: Instead of having to reason about an entanglement of complex protocols,
one can devise several modules with the property that, when taken in isolation,
each module refines the SLin I/O-automaton. Hence complex protocols can be
divided into smaller modules that can be analyzed independently of each other.
In particular, it allows to optimize an existing protocol by creating separate
optimization modules, prove each optimization correct in isolation, and obtain
the correctness of the overall protocol from the correctness of the existing
one.

In this document we define the SLin I/O-automaton and prove the Composition
Theorem, which states that the composition of two instances of the SLin
I/O-automaton behaves as a single instance of the SLin I/O-automaton. We use a
refinement mapping to establish this fact.


%%%%%%%%%%%%%%%%%%%%%%%%%%%%%%
% <session>
  % \input{session}
  \input{UPF-Firewall}
    \input{NetworkModels}
      \input{NetworkCore}
      \input{DatatypeAddress}
      \input{DatatypePort}
      \input{IntegerAddress}
      \input{IntegerPort}
      \input{IntegerPort_TCPUDP}
      \input{IPv4}
      \input{IPv4_TCPUDP.tex}
    \input{PacketFilter.tex}
      \input{PolicyCore}
      \input{PolicyCombinators}
      \input{PortCombinators}
      \input{ProtocolPortCombinators}
      \input{Ports}
      \input{NAT}
    \input{FWNormalisation.tex}
      \input{FWNormalisationCore.tex}
      \input{NormalisationGenericProofs.tex}
      \input{NormalisationIntegerPortProof.tex}
      \input{NormalisationIPPProofs.tex}
    \input{StatefulFW}
      \input{StatefulCore}
      \input{FTP}
      \input{FTP_WithPolicy}
      \input{VOIP}
      \input{FTPVOIP}
  %%%%%%%%%%%%%%%%%%%%%%%%%%%%%
  \input{Examples.tex}
    \input{DMZ.tex}
      \input{DMZDatatype.tex}
      \input{DMZInteger.tex}
    \input{PersonalFirewall.tex}
      \input{PersonalFirewallInt.tex}
      \input{PersonalFirewallIpv4.tex}
      \input{PersonalFirewallDatatype.tex}
    \input{Transformation.tex}
      \input{Transformation01.tex}
      \input{Transformation02.tex}
    \input{NAT-FW.tex}
    \input{Voice_over_IP.tex}
% </session>
%%%%%%%%%%%%%%%%%%%%%%%%%%%%%%

\bibliographystyle{abbrvnat}
\bibliography{root}
\end{document}

%%% Local Variables:
%%% mode: latex
%%% TeX-master: t
%%% End:
