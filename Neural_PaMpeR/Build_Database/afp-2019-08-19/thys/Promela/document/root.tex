\documentclass[11pt,a4paper]{article}
\usepackage{isabelle,isabellesym}
\usepackage[english]{babel}
\usepackage{amssymb}
\usepackage[only,bigsqcap]{stmaryrd}

% this should be the last package used
\usepackage{pdfsetup}

% urls in roman style, theory text in math-similar italics
\urlstyle{rm}
\isabellestyle{it}

\newcommand{\isaheader}[1]{#1}
\newcommand{\isachapter}[1]{\chapter{#1}}
\newcommand{\isasection}[1]{\section{#1}}

% General
\newcommand{\ie}{i.\,e.\ }
\newcommand{\eg}{e.\,g.\ }
\newcommand{\wrt}{w.\,r.\,t.\ }
\newcommand{\cf}{cf.\ }

\begin{document}

\title{Promela Formalization}
\author{By Ren\'{e} Neumann}
\maketitle

\begin{abstract}
    We present an executable formalization of the language Promela, the description language for models of the model checker SPIN. This formalization is part of the work for a completely verified
    model checker (CAVA), but also serves as a useful (and executable!) description of the semantics of the language itself, something that is currently missing.
    The formalization uses three steps: It takes an abstract syntax tree generated from an SML parser, removes syntactic sugar and enriches it with type information. This further gets translated into a transition system, on which the semantic engine (read: successor function) operates.
\end{abstract}

\clearpage

\tableofcontents

\clearpage

\chapter{Introduction}
  This development provides an efficient, extensible, machine checked collections framework for use
  in Isabelle/HOL. The library adopts the concepts of interface, implementation and generic algorithm
  known from object oriented (collection) libraries like the C++ Standard Template Library\cite{C++STL} or 
  the Java Collections Framework\cite{JavaCollFr} and makes them available in the Isabelle/HOL environment.

  The library uses data refinement techniques to refine an abstract specification (in terms of high-level concepts such as sets) to a more concrete implementation (based on collection datastructures like red-black-trees).
  This allows algorithms to be proven on the abstract level at which proofs are simpler because they are not cluttered with low-level details.

  The code-generator of Isabelle/HOL can be used to generate efficient code in all supported target languages, i.e. Haskell, SML, and OCaml.

  For more documentation and introductory material refer to the userguide (Section~\ref{thy:Userguide}) and the ITP-2010 paper \cite{LammichLochbihler2010ITP}.


% sane default for proof documents
\parindent 0pt\parskip 0.5ex

% generated text of all theories
\input{session}

% optional bibliography
\bibliographystyle{abbrv}
\bibliography{root}

\end{document}

%%% Local Variables:
%%% mode: latex
%%% TeX-master: t
%%% End:
