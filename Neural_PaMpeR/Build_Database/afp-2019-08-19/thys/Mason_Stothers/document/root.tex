\documentclass[11pt,a4paper]{article}
\usepackage{isabelle,isabellesym}

% further packages required for unusual symbols (see also
% isabellesym.sty), use only when needed

%\usepackage{amssymb}
  %for \<leadsto>, \<box>, \<diamond>, \<sqsupset>, \<mho>, \<Join>,
  %\<lhd>, \<lesssim>, \<greatersim>, \<lessapprox>, \<greaterapprox>,
  %\<triangleq>, \<yen>, \<lozenge>

%\usepackage{eurosym}
  %for \<euro>

%\usepackage[only,bigsqcap]{stmaryrd}
  %for \<Sqinter>

%\usepackage{eufrak}
  %for \<AA> ... \<ZZ>, \<aa> ... \<zz> (also included in amssymb)

%\usepackage{textcomp}
  %for \<onequarter>, \<onehalf>, \<threequarters>, \<degree>, \<cent>,
  %\<currency>

% this should be the last package used
\usepackage{pdfsetup}

% urls in roman style, theory text in math-similar italics
\urlstyle{rm}
\isabellestyle{it}

% for uniform font size
%\renewcommand{\isastyle}{\isastyleminor}


\begin{document}

\title{The Mason--Stothers theorem}
\author{Manuel Eberl}
\maketitle

\begin{abstract}
This article provides a formalisation of Snyder's simple and elegant proof of the Mason--Stothers theorem~\cite{snyder2000,lemmermeyer05}, which is the polynomial analogue of the famous $abc$ Conjecture for integers. Remarkably, Snyder found this very elegant proof when he was still a high-school student.

In short, the statement of the theorem is that three non-zero coprime polynomials $A$, $B$, $C$ over a field which sum to $0$ and do not all have vanishing derivatives fulfil $\textrm{max}\{\textrm{deg}(A),\textrm{deg}(B),\textrm{deg}(C)\} < \textrm{deg}(\textrm{rad}(ABC))$ where $\textrm{rad}(P)$ denotes the \emph{radical} of $P$, i.\,e.\ the product of all unique irreducible factors of $P$.

This theorem also implies a kind of polynomial analogue of Fermat's Last Theorem for polynomials: except for trivial cases, $A ^ n + B ^ n + C ^ n = 0$ implies $n \leq 2$ for coprime polynomials $A$, $B$, $C$ over a field.
\end{abstract}

\tableofcontents
\newpage

% sane default for proof documents
\parindent 0pt\parskip 0.5ex

% generated text of all theories
\input{session}

% optional bibliography
\newpage
\bibliographystyle{abbrv}
\bibliography{root}

\end{document}

%%% Local Variables:
%%% mode: latex
%%% TeX-master: t
%%% End:
