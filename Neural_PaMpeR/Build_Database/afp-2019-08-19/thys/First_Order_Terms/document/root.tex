\documentclass[11pt,a4paper]{article}
\usepackage{isabelle,isabellesym}
\usepackage{amssymb}
\usepackage{amsmath}
\usepackage{xspace}

% this should be the last package used
\usepackage{pdfsetup}

% urls in roman style, theory text in math-similar italics
\urlstyle{rm}
\isabellestyle{it}

\newcommand\isafor{\textsf{IsaFoR}}
\newcommand\ceta{\textsf{Ce\kern-.18emT\kern-.18emA}}

% for uniform font size
%\renewcommand{\isastyle}{\isastyleminor}

\begin{document}

\title{First-Order Terms\footnote{Supported by FWF (Austrian Science Fund) projects Y757 and P27502}}
\author{Christian Sternagel \and Ren\'e Thiemann}
\maketitle


\begin{abstract}
We formalize basic results on first-order terms, including a first-order unification algorithm,
as well as well-foundedness of the subsumption order. 
This entry is part of the \emph{Isabelle Formalization of Rewriting} \isafor~\cite{isafor},
where first-order terms are omni-present: the unification algorithm 
is used to certify several confluence and termination techniques, like
critical-pair computation and dependency graph approximations; and the subsumption order
is a crucial ingredient for completion.
\end{abstract}

\tableofcontents

\section{Introduction}

We define first-order terms, substitutions, the subsumption order, and a unification algorithm.
In all these definitions type-parameters are used to specify variables
and function symbols, but there is no explicit signature. 

The unification algorithm has been formalized following a textbook on term rewriting~\cite{AllThat}.

The complete \isafor\ library is available at:
\begin{quote}
\url{http://cl-informatik.uibk.ac.at/isafor/}
\end{quote}

% sane default for proof documents
\parindent 0pt\parskip 0.5ex

% generated text of all theories
\input{session}

% optional bibliography
\bibliographystyle{abbrv}
\bibliography{root}

\end{document}

%%% Local Variables:
%%% mode: latex
%%% TeX-master: t
%%% End:
