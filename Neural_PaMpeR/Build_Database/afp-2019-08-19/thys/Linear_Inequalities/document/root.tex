\documentclass[11pt,a4paper]{article}
\usepackage{isabelle,isabellesym}

\usepackage{amsmath}
\usepackage{amssymb}
\usepackage{amsthm}
\usepackage{authblk}



\newcommand\ints{\mathbb{Z}}
\newcommand\reals{\mathbb{R}}

\newtheorem{theorem}{Theorem}
% this should be the last package used
\usepackage{pdfsetup}

% urls in roman style, theory text in math-similar italics
\urlstyle{rm}
\isabellestyle{it}

\newcommand\isafor{\textsf{IsaFoR}}
\newcommand\ceta{\textsf{Ce\kern-.18emT\kern-.18emA}}

\begin{document}

\title{Linear Inequalities\footnote{Supported by FWF (Austrian Science Fund) project Y757.}}
\author[1]{Ralph Bottesch}
\author[2]{Alban Reynaud}
\author[1]{Ren\'e Thiemann}
\affil[1]{University of Innsbruck}
\affil[2]{ENS Lyon}

\maketitle

\begin{abstract}
  We formalize results about linear inqualities, mainly from Schrijver's book
  \cite{schrijver1998theory}.
  The main results are the proof of the fundamental theorem on linear
  inequalities, Farkas' lemma, Carath\'eodory's theorem, 
  the Farkas-Minkowsky-Weyl theorem, 
  the decomposition theorem of
  polyhedra, and Meyer's result that the integer hull of a polyhedron is a
  polyhedron itself. Several theorems include bounds on the appearing
  numbers, and in particular we provide an a-priori bound
  on mixed-integer solutions of linear inequalities. 

\end{abstract}

\tableofcontents

\section{Introduction}

The motivation for this formalization is the aim of developing a verified theory solver
for linear integer arithmetic. Such a solver can be a combination of a simplex-implementation within
a branch-and-bound approach, that might also utilize Gomory cuts \cite[Section 4 of the extended version]{incremental_simplex}. 
However, the branch-and-bound algorithm does
not terminate in general, since the search space in infinite. To solve this latter problem, one can use
results of Papadimitriou: he showed that whenever a set of linear inequalities has an integer
solution, then it also has a small solution, where the bound on such a solution can be computed
easily from the input \cite{Papad}.

In this entry, we therefore formalize several results on linear inequalities which are required
to obtain the desired bound, by following the proofs of Schrijver's 
textbook~\cite[Sections 7 and 16]{schrijver1998theory}.

We start with basic definitions and results on cones, convex hulls, and polyhedra.
Next, we verify the fundamental theorem of linear inequalities, which in
our formalization shows the equivalence of four statements to describe a cone. 
From this theorem, one easily derives Farkas' Lemma and Carath\'eodory's theorem.
Moreover we verify the Farkas-Minkowsky-Weyl theorem, that a convex cone is polyhedral if and only if
it is finitely generated, and use this result to obtain 
the decomposition theorem for polyhedra, i.e., that a polyhedron can always be decomposed into
a polytope and a finitely generated cone. 
For most of the previously mentioned results, we include bounds, so that in particular we have a
quantitative version of the decomposition theorem, which provides bounds on the vectors that construct
the polytope and the cone, and where these bounds are 
computed directly from the input polyhedron that should be decomposed.

We further prove the decomposition theorem also for the integer hull of a polyhedron, using the same bounds,
which gives rise to small integer solutions for linear inequalities. We finally formalize a direct proof
for the more general case of mixed integer solutions, where we also permit both strict and non-strict 
linear inequalities. 

\begin{theorem}
\label{thm}
Consider  
$A_1 \in \ints^{m_1 \times n}, b_1 \in \ints^{m_1}, A_2 \in \ints^{m_2 \times n}, b_2 \in \ints^{m_2}$.
Let $\beta$ be a bound on $A_1,b_1,A_2,b_2$, i.e., $\beta \geq |z|$ for all numbers $z$ that occur within $A_1,b_1,A_2,b_2$. Let $n = n_1+n_2$.
Then if $x \in \ints^{n_1} \times \reals^{n_2} \subseteq \reals^n$ is a mixed integer solution of the linear inequalities, i.e., 
$A_1x \leq b_1$ and $A_2x < b_2$, then there also exists a mixed integer solution 
$y \in \ints^{n_1} \times \reals^{n_2}$ where 
$|y_i| \leq (n+1)! \cdot \beta^n$ for each entry $y_i$ of $y$.
\end{theorem}

The verified bound in Theorem~\ref{thm} in particular implies that integer-satisfiability
of linear-inqualities with integer coefficients is in NP.

% include generated text of all theories
\input{session}



\bibliographystyle{abbrv}
\bibliography{root}

\end{document}
