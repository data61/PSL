\documentclass[11pt,a4paper]{article}
\usepackage{isabelle,isabellesym}

\usepackage{amsmath}
\usepackage{amssymb}
\usepackage{amsthm}
\usepackage{xspace}
\usepackage[utf8]{inputenc}

% this should be the last package used
\usepackage{pdfsetup}

% urls in roman style, theory text in math-similar italics
\urlstyle{rm}
\isabellestyle{it}

\newtheorem{theorem}{Theorem}%[section]
\newtheorem{corollary}{Corollary}%[section]
\newcommand\rats{\mathbb{Q}}
\newcommand\ints{\mathbb{Z}}
\newcommand\reals{\mathbb{R}}
\newcommand\complex{\mathbb{C}}

\begin{document}

\title{Irrational Rapidly Convergent Series}
\author{Angeliki Koutsoukou-Argyraki and Wenda Li}
\maketitle

\begin{abstract}
We formalize with Isabelle/HOL a proof of a theorem by J. Han\v cl asserting the irrationality of the sum of a series
consisting of rational numbers, built up by sequences that fulfill certain properties.
Even though the criterion is a number theoretic result, the proof makes use only of analytical arguments. 
We also formalize a corollary of the theorem for a specific series fulfilling the assumptions of the theorem.
\end{abstract}

\tableofcontents

\section{Main Theorem and Sketch of the Proof}
We formalize the proof of the following theorem by J. Han\v cl (Theorem 3 in \cite{hancl}) :
\begin{theorem}(Theorem 3  in \cite{hancl})
	Let  $A \in \mathbb{R}$ with $A>1$. Let $\{d_n \}^{\infty}_{n=1}  \in \mathbb{R}$ with $d_n >1$ for all $n \in \mathbb{N}$. Let $\{a_n \}^{\infty}_{n=1}  \in \mathbb{Z}^+$,
	$\{b_n \}^{\infty}_{n=1}  \in \mathbb{Z}^+$ such that :
	$$(1)~\lim_{n \rightarrow \infty} a_n^{\frac{1}{2^n}} = A , $$
	for all sufficiently  large $n \in \mathbb{N}$ :
	$$(2)~\frac{A}{ a_n^{\frac{1}{2^n}}    }  > \prod^{\infty}_{j=n} d_j$$
	and
	$$(3)~\lim_{n \rightarrow \infty}\frac{d_n^{2^n}}{b_n} =\infty.  $$
	Then the series $\alpha = \sum^{\infty}_{n=1} \frac{b_n}{a_n}$ is an irrational number.
\end{theorem}

The first step is to show that the series $ \sum^{\infty}_{n=1} \frac{b_n}{a_n}$ converges to some $\alpha \in \mathbb{R}$.
To show that $\alpha \in  \mathbb{R} \setminus \mathbb{Q}$ we argue by proof by contradiction (to this end several auxiliary lemmas are firstly shown).
In particular, assuming that $\alpha \in \mathbb{Q}$, i.e. that there exist $p, q \in \mathbb{Z}^+$ such that $\alpha = \frac{p}{q}$, we show that
a quantity $\mathcal{A}(n) \geq 1$ for all $n \in \mathbb{N}$.
At the same time, we find $n \in \mathbb{N}$ for which  $\mathcal{A}(n) < 1$, yielding a contradiction from which we deduce the irrationality of the sum of the series.
\\ \\
For the proof see \cite{hancl}. We note that the proof involves only elementary Analysis (criteria for convergence/divergence for sequences and series and several inequalities)
and not any arithmetical/number theoretic arguments. Obviously for the formal proof we had to make many intermediate arguments explicit.  Proofs of length of roughly 2 A4 pages
in the original paper by J. Han\v cl  were formalized in almost 1100 lines of code.

\section{Corollary}
We moreover formalize the following corollary that asserts the irrationality of the sum of an instance of a series that fulfills the assumptions of the theorem :

\begin{corollary} (Corollary 2 in \cite{hancl}) Let  $A \in \mathbb{R}$ with $A>1$. Let  $\{a_n \}^{\infty}_{n=1}  \in \mathbb{Z}^+$,
	$\{b_n \}^{\infty}_{n=1}  \in \mathbb{Z}^+$ such that :
	$$\lim_{n \rightarrow \infty} a_n^{\frac{1}{2^n}} = A  $$ and for all sufficiently  large $n \in \mathbb{N}$
	(in particular: for $n \geq 6$)
	$$a_n^{\frac{1}{2^n}} (1+ 4 (2/3)^n)  \leq A $$ and  $$b_n \leq 2^{(4/3)^{n-1}} .$$
	Then the series $\sum^{\infty}_{n=1} \frac{b_n}{a_n}$ is an irrational number.
\end{corollary}
The above corollary is an immediate consequence of the theorem by setting 
$d_n = 1 + (2/3)^n$. For the formalized proof of the corollary one more auxiliary lemma was required.

\input{session}

\section{Acknowledgements}
A. K.-A. and W.L. were supported by the ERC Advanced Grant ALEXANDRIA (Project 742178) funded by the European Research Council and led by Professor Lawrence Paulson at the University of Cambridge, UK.


\bibliographystyle{abbrv}
\bibliography{root}

\end{document}
